\subsection{Test Case}
A test case is designed to help find out whether a design system created for a common purpose, in this case a specific type of web application, helps users work better with an application. To get better test results, in this elaboration one test case is performed with two test applications. One with SDS implemented and one without. \\
Since both systems are visually identical, it is not possible to test both systems with the same person. The tester would already know how to operate and navigate through the test system by the second run. Therefore, it is necessary to perform a reasonable number of test runs on both systems with approximately the same type of person. By person type in this elaboration is meant what a person does in his daily job, e.g. accountant.  \\
The test provides two results for this elaboration. First, the test execution that each person does is observed. The purpose of this observation is to see how the subject responds to the navigation structure of the web page. How the subject navigates to the next page? Which controls he/she uses, whether mouse, keyboard or other special controls? Testing is done in person to also capture emotions and reactions to the appearance and disappearance of content. At the end of a test run, the subject is asked for additional feedback on the application just tested. With this subjective input, it is possible to assess whether the use of \ac{SDS} affects anything. \\
In addition, time measurement points are implemented in each application. These points provide objective feedback about the time spent on certain views or actions. The measurement points are located in the same place so that a comparison between both application types is possible.  In this way, it is feasible to obtain unbiased data for evaluation. \\
In preparation for the test runs, the two applications with and without implementation of the design system are published as static web pages. The test person opens one of the two applications and finds a description of the test procedure. The description informs the user that the following application might implement a design system. After the brief introduction, three test steps are explained to the user:

\begin{enumerate}
    \item Navigate to the data table containing the records of different companies.
    \item Find the action to create a new data record.
    \item Fill out the form and submit a new entry.
\end{enumerate}

Since users are not given any instructions on how to navigate to the data table or with what details to fill in the new record, users are free to explore the demo application. The test steps are intentionally kept very vague to give users the freedom to develop their own patterns in their interaction. \\
After reading all the instructions, you will be asked to start the test by clicking the "Start" button below. This will trigger the first measurement point, which is the start of the demo. \\
This will take users to the landing page shown in Figure \ref{landing_page}. On this page, users need to figure out how to navigate to the companies' data table. By scanning the view, they should figure out how to use the navigation bar at the top to navigate to the Companies page. When the user clicks on the navigation bar, they will be taken to the data table view and the second measurement point will be triggered. The measured time needed is stored in the memory. \\
In the second view (Figure \ref{data_table}), the user sees the data table with the entries contained in it. To add a new entry to the data table, the user must click on the "Add +" button in the upper right corner of this view. By pressing the button, the view changes to the third view (Figure \ref{adding_form}). As with the last view change, a new measurement point is triggered and the time data is stored in memory. \\ 
In the last view, the user fills in the given form and clicks on "Save" to store the data in the table. The action again triggers and saves a measurement point for the test. When saving the new data unit in the table, the user is navigated back to the table and automatically downloads a JSON file containing all the measurement points just described. This JSON file is then sent to the author of this elaboration to collect the measurement of each participant. \\

By collecting all the measurement data and observing the user behavior during the interaction, this test case provides a lot of data for evaluation. This concludes this chapter and leads to the next chapter where the test results are presented and evaluated.