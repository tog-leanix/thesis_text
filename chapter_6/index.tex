\newpage
\section{Conclusion}
This elaboration presents a possibility to perform standardization differently. A design system represents a way of creating user interfaces for ergonomically correct views. With today's technology and standardized browsers, finding a central solution that allows developers to create reusable components without the application restricted to a particular framework is no problem.  \\
Chapter \ref{saas_design_system} presents a possible concept for implementing a design system. The \acl{SDS} allows developers and designers to focus more on delivering value to their users because they do not have to worry about accessibility or best practices. In addition, the onboarding time for creators is short because a design system provides enough tools to get them started quickly. \\
With all the tools, developers can build their design system based on the \ac{SDS}. It gives them a blueprint for a design system. Not only does it extend components through the interfaces provided, but it also gives developers the ability to improve the existing components. The \ac{SDS} leaves enough room for creativity for their \ac{SaaS} products and brand.\\
A vital feature of the \ac{SDS} is the community aspect that surrounds it. Without it, the \ac{SDS} would be just another component library. Nevertheless, providing principles, guidelines, and a platform for designers and developers makes \ac{SDS} special. These community-built values reflect the needs of developers and designers much better than an ISO standard that is updated every two years. In addition, quick response time is essential in a fast-paced world. \\
The results of the test of the \ac{SDS} in Chapter \ref{test_results} show a specific result. Efficiency for the end users of SaaS products is crucial when developing a design system like SDS. The test result shows that a design system helps maintain an application's consistency. The visual design of both test applications is the same. Even though the observation seems to show no difference between the two applications and the data collected supports this observation. There is no statistically significant difference between the total time required for the task and the version used in the test. However, the graphs shown indicate a slight trend that the \ac{SDS} supports the consistency of the applications. Consistency is an interesting point to follow. \\
In summary, the \ac{SDS} is a valid concept for a different kind of user interface standardization for \ac{SaaS} products. Not only is it much easier to implement in existing systems, but it also provides added value for end users. Of course, this elaboration is just a concept of what such a system could look like, and much more work is needed in a production environment to create a fully functional system. Nevertheless, it might be interesting to investigate these results as a first proposal further. 
\subsection{Outlook}