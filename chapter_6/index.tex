\newpage
\section{Conclusion}
This elaboration presents a possibility to perform standardization in a different way. A design system represents a way of creating user interfaces for ergonomically correct views. With today's technology and standardized browsers, it is not a problem anymore to find a central solution that allows developers to create reusable components without the application restricting to a certain framework. \\
Chapter \ref{saas_design_system} presents a possible concept for implementing a design system. The \acl{SDS} allows developers and designers to focus more on delivering value to their users because they don't have to worry about accessibility or best practices. The onboarding time for creators is short because a design system provides enough tools to get them started quickly. \\
With all the tools in hand, developers can even build their own design system based on the \ac{SDS}. It gives them a blueprint for a design system. Not only does it extend components through the interfaces provided, but it also gives developers the ability to manipulate the existing components. This leaves enough room for creativity for the own \ac{SaaS} products and their brand.\\
A key feature of the \ac{SDS} is the community aspect that surrounds it. Without it, the \ac{SDS} could be just another component library. But providing principles, guidelines, and a platform for designers and developers is what makes \ac{SDS} special. These community-built values reflect the needs of developers and designers much better than an ISO standard that is updated every two years. In a fast-paced world, a quick response time is important.\\
The results of the test of the \ac{SDS} in Chapter \ref{test_results} show a clear result. Efficiency for the users of \ac{SaaS} products is the crucial point when developing a design system like \ac{SDS}. The test result shows that a design system helps to maintain the consistency of an application. Both test applications are visually designed in the same way. Even though the observation seems to show no difference between the two applications, the data collected shows a different pattern. The test shows that users were able to complete their tasks 20\% faster with the \ac{SDS} application. \\
In summary, the \ac{SDS} is a valid concept for a different kind of user interface standardization for \ac{SaaS} products. Not only is it much easier to implement in existing systems, but it also provides added value for end users. Of course, this elaboration is just a concept of what such a system could look like, and much more work is needed in the real world to create a fully functional system. However, as a first proposal, it might be interesting to follow up on these results.
\subsection{Outlook}