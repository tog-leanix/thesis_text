\subsection{Outlook}
A possible extension of this elaboration is to adopt the idea of \ac{SDS}. Introduce new components and explore patterns. Bring this design system to life on a platform and provide a social platform for sharing current design system ideas. It is important to bring developers and designers together to work on this idea. The only driving force for this design system is community. It is the chance to simplify the standardization. \\
Of course, not all topics of such a design system can be covered in one elaboration. There are still many open questions. For example, what would a development process in such an open source project look like? There are many examples of this in open source projects, but mostly for code projects. But what about the non-code content like guidelines and design principles. Who will make those decisions? Because one person should not be the only one to determine such decisions. \\
In addition, the \ac{SDS} is not bulletproof in terms of its data. The testing was done with a relatively small group to get qualitative feedback as well. A next study would expand this to a field test with nearly hundreds of participants. This would provide better insight into the actual performance of the \ac{SDS}. In addition, with a more advanced component, it would also be possible to test different test cases. Examples would include modal interaction, user profile editing, and data deletion. \\
And just as big papers are not the Holy Grail for standardization, the same could be true for \ac{SDS}. So the question remains, how can standardization work better in the age of SaaS products for designers, developers and their end users? And yet another solution might seem better. Right now, design systems are on the rise to standardize user interfaces in applications within organizations, and this elaboration shows that it is also possible to standardize products across an entire domain of products. 