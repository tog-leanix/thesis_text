\subsection{Outlook}
A possible extension of this elaboration is to adopt the idea of \ac{SDS}. Introduce new components and explore patterns. Bring this design system to life on a platform and provide a social platform for sharing existing design system ideas. It is essential to bring developers and designers together to work on this idea. The only driving force for this design system is the community building it. It is the chance to simplify the standardization. \\
Of course, the elaboration does not cover all topics of such a design system. There are still many open questions. For example, what would be the development process in such an open source design system? There are many examples of this in open source projects, but primarily for code projects. But what about the non-code content like guidelines and design principles. Who will make those decisions? Because one person should not be the only one to determine such decisions. \\
In addition, the tests performed show no increase in daily tasks, but the consistency is worth exploring further. The testing was done with a relatively small group to get qualitative feedback. The following study would expand this to a field test with nearly hundreds of participants. The results would provide better insight into other usage patterns using the \ac{SDS}. Furthermore, with more advanced components, it would also be possible to test different use cases showing a performance difference. Such use cases include modal interactions, user profile editing, and data deletion. \\
Furthermore, just as big papers are not the Holy Grail for standardization, the same could be true for \ac{SDS}. So the question remains, how can standardization work better in the age of \ac{SaaS} products for designers, developers, and end users? However, another solution might seem better. Right now, design systems are rising to standardize user interfaces in applications within organizations. This elaboration shows that it is also possible to standardize products across an entire domain of products. 