\documentclass[paper=a4,11pt,parskip=half,toc=listof]{scrartcl}
\usepackage{etoolbox}
\newtoggle{german}

% % % % % LANGUAGE % % % %
% Make your choice here
\togglefalse{german} % English
%\toggletrue{german} % German
% % % % % \LANGUAGE % % % %

\iftoggle{german}{
\usepackage[ngerman]{babel} % Deutsche Sprachanpassungen
\usepackage[T1]{fontenc}    % Silbentrennung bei Sonderzeichen
\usepackage[utf8]{inputenc} % Direkte Angabe von Umlauten im Dokument.Wenn Sie an einem Mac sitzen,verwenden Sie ggf. „macce“ anstatt „utf8“.
\usepackage[autostyle=true,german=quotes]{csquotes} % Anfuehrungszeichen\
}{
\usepackage[utf8]{inputenc}}
\usepackage{enumitem}
\usepackage{scrpage2}
\usepackage{listings}
\usepackage{color}
\usepackage{textcomp}
\usepackage{mathtools}
\usepackage{nicefrac}
\usepackage{amsmath}
\usepackage[T1]{fontenc}
\usepackage[german]{translator}
\usepackage{pgfgantt}
\usepackage{graphicx}
\usepackage{longtable}
\usepackage{float}
\usepackage{setspace}
\usepackage{url}
\usepackage{tabularx}
\usepackage{makecell}
\usepackage{multirow}
\usepackage{wrapfig}
\usepackage{pdfpages}
\usepackage{amssymb}
\usepackage{graphicx}
\usepackage[font=small]{caption} % very thin
\usepackage{subcaption}
\usepackage{picinpar}
\usepackage{multicol}
\usepackage{tcolorbox}
\usepackage{rotating}

%Bibliography
\usepackage[
backend=biber,
style=ieee,
sortlocale=en_GB,
natbib=true,
url=false, 
doi=true,
isbn=true,
eprint=false
]{biblatex}


\addbibresource{references.bib}
% No footnotes on the next page please
\interfootnotelinepenalty=10000
\lstdefinelanguage{TypeScript}{
  keywords={typeof, new, true, false, catch, function, return, null, catch, switch, var, if, in, while, do, else, case, break, const, let, interface, type, extends},
  keywordstyle=\color{blue}\bfseries,
  ndkeywords={class, export, boolean, throw, implements, import, this},
  ndkeywordstyle=\color{darkgray}\bfseries,
  identifierstyle=\color{black}\ttfamily,
  sensitive=false,
  comment=[l]{//},
  morecomment=[s]{/*}{*/},
  commentstyle=\color{purple}\ttfamily,
  stringstyle=\color{red}\ttfamily,
  morestring=[b]',
  morestring=[b]",
  morestring=[b]`
}

\lstset{ 
  language=TypeScript,                 % the language of the code
  backgroundcolor=\color{white},   % choose the background color; you must add \usepackage{color} or \usepackage{xcolor}; should come as last argument
  basicstyle=\footnotesize,        % the size of the fonts that are used for the code
  breakatwhitespace=false,         % sets if automatic breaks should only happen at whitespace
  breaklines=true,                 % sets automatic line breaking
  captionpos=b,                    % sets the caption-position to bottom
  extendedchars=true,              % lets you use non-ASCII characters; for 8-bits encodings only, does not work with UTF-8
  frame=single,	                   % adds a frame around the code
  keepspaces=true,                 % keeps spaces in text, useful for keeping indentation of code (possibly needs columns=flexible)
  keywordstyle=\color{blue},       % keyword style
  numbers=left,                    % where to put the line-numbers; possible values are (none, left, right)
  numbersep=5pt,                   % how far the line-numbers are from the code
  numberstyle=\color{gray}\bfseries, % the style that is used for the line-numbers
  rulecolor=\color{black},         % if not set, the frame-color may be changed on line-breaks within not-black text (e.g. comments (green here))
  showspaces=false,                % show spaces everywhere adding particular underscores; it overrides 'showstringspaces'
  showstringspaces=false,          % underline spaces within strings only
  showtabs=false,                  % show tabs within strings adding particular underscores
  stepnumber=1,                    % the step between two line-numbers. If it's 1, each line will be numbered
  tabsize=2,	                   % sets default tabsize to 2 spaces
  title=\lstname                   % show the filename of files included with \lstinputlisting; also try caption instead of title
}
%opening

% Introduction of a design system to establish a frontend engineering and design platform
% Creation of a front-end platform by introducing a design system
% Introduction of a design system - A frontend engineering and design platform
\def\ThesisTitle{User interface standardization of \acl{SaaS} products}
\def\ThesisAuthor{Tom Gehder}
\def\ThesisLocation{Sankt Augustin}
\def\ThesisType{Master}
\def\CourseType{Computer Science}
\def\ThesisPubDate{July 28, 2022} % Change here to the date you are going to print your thesis
\def\ThesisFirstSupervisor{Prof. Dr. Manfred Kaul}
\def\ThesisSecondSupervisor{Prof. Dr. Sascha Alda}
\def\ThesisExternalSupervisor{-}
\def\ThesisExternalCompany{-}
\def\ThesisSubject{}


\usepackage{todonotes}
\usepackage{pdfpages} % directly include pdf pages
\usepackage{algorithmic} % pseudo-code
\usepackage{blindtext}
\usepackage[printonlyused]{acronym} 
%\usepackage[firstpage]{draftwatermark} % comment in if you submit a draf. 
\DeclarePairedDelimiter{\ceil}{\lceil}{\rceil}

% new types for a table
\newcolumntype{C}[1]{>{\centering\arraybackslash}p{#1}}
\newcommand{\specialcell}[2][l]{%
  \begin{tabular}[#1]{@{}c@{}}#2\end{tabular}}
  



%%%% Font %%%%
% \usepackage{times} % times in text
\usepackage{mathptmx} % times in math
\usepackage{setspace} \onehalfspacing %
\usepackage[paper=a4paper]{geometry}
\setlength{\parindent}{0pt} % no indent
\setlength{\headheight}{1.1\baselineskip}
\setcounter{tocdepth}{4}
\setcounter{secnumdepth}{4}
%%%% Font %%%%

%%%% footer and header %%%%
\usepackage{scrpage2}%
\pagestyle{scrheadings}%  S
\clearscrheadfoot% 
\setheadwidth{text}%
\automark{section}% 
\ohead{\textbf{\pagemark}}
\renewcommand{\sectionmark}[1]{\markright{\ #1}} 
\ihead{\textbf{\rightmark}}
\setheadsepline{0.5pt}
%%%% \footer and header %%%%

\newcolumntype{Y}{>{\centering\arraybackslash}X}




\usepackage{hyperref}
\hypersetup{
    unicode=false,          % non-Latin characters in Acrobat’s bookmarks
    pdftoolbar=true,        % show Acrobat’s toolbar?
    pdfmenubar=true,        % show Acrobat’s menu?
    pdffitwindow=false,     % window fit to page when opened
    pdfstartview={FitH},    % fits the width of the page to the window
    pdftitle={\ThesisTitle},    % title
    pdfauthor={\ThesisAuthor},     % author
    pdfsubject={\ThesisTitle},   % subject of the document
    pdfcreator={\ThesisAuthor},   % creator of the document
    pdfproducer={\ThesisAuthor}, % producer of the document
    pdfkeywords={802.11} {DCF} {long-distance} {modeling}, % list of keywords
    pdfnewwindow=true,      % links in new window
    colorlinks=false,       % false: boxed links; true: colored links
    linkcolor=red,          % color of internal links (change box color with linkbordercolor)
    citecolor=green,        % color of links to bibliography
    filecolor=magenta,      % color of file links
    urlcolor=cyan           % color of external links
}


\begin{document}
%%%%%%%%%%%%%%%%%%%%% Startseite %%%%%%%%%%%%%%%%%%%%%%%%%%%
\include{titlepage}
\newgeometry{top=4cm, bottom=3cm, left=4.5cm, right=3cm}
Name: Tom Gehder \\
Adresse: Weberstraße 79, 53113 Bonn\\

\section*{Erklärung}
Hiermit erkläre ich an Eides Statt, dass ich die vorliegende Arbeit selbst angefertigt
habe; die aus fremden Quellen direkt oder indirekt übernommenen Gedanken sind
als solche kenntlich gemacht.\\

Die Arbeit wurde bisher keiner Prüfungsbehörde vorgelegt und auch noch nicht
veröffentlicht.


\begin{table}[b]
    \begin{tabular}{p{0.5\linewidth} p{0.5\linewidth}}
        Unterschrift & Sankt Augustin, den \\
        \hline
    \end{tabular}
\end{table}

\pagenumbering{Roman} % Big roman numbers
\addtocontents{toc}{\protect\markright{}}

\newpage
\tableofcontents
\cleardoublepage
\pagenumbering{arabic}
\newpage
\listoftables % add list of tables
\listoffigures % add list of figures 
\lstlistoflistings % add list of listings 
\section*{List of Abbreviations}
\begin{acronym}
    \acro{SDS}[SDS]{SaaS Design System}
\end{acronym} % include the acronym thing

\setcounter{page}{1}
\newpage
%%%%%%%%%%%%%%%%%%%%% Startseite %%%%%%%%%%%%%%%%%%%%%%%%%%%
\setcounter{tocdepth}{4} 
\setcounter{secnumdepth}{4}

\pagenumbering{arabic} % arabic numbers for the main part

% Here the chapters are included

\section{Introduction}
Standardization in technology is an essential topic of great significance. Without standardization, there would be no progress in computer science today. \\
A look at the past shows that inventors made the first attempt at standardization in the 18th century. During the industrial revolution, standards were urgently needed to align and reliably connect machines. Thus, engineers created the first known standard with the screw and nut. With this invention, technology started to spread and develop rapidly during this time.  \cite{wiki_standardization_2022} \\
Today, no one thinks about the specification of a bolt or a nut when they use it. Standardization aims to reduce the demand for specific intstructions and increase time for critical tasks. \\

\subsection{\acl{SaaS}}

Speaking of technology, a new star has risen in recent years, namely \acl{SaaS}. \acl{SaaS} is a discipline of cloud computing. In general, \ac{SaaS} applications are hosted in the cloud, with which users can interact via the Internet. In this way, it is much easier to provide users with services that previously required them to install software on their desktops. This type of software often comes with a subscription model that allows users to switch from one software to another within a month. \\
\ac{SaaS} providers take care of all kinds of infrastructure issues, so the end user doesn't have to worry about those issues. In return, the end user pays for the entire service and not just for the actual application. \\
Since this type of software is available over the Internet, any connected device can access the software through a web browser. The buzzword Web 2.0 describes the evolution away from on-premise software to services delivered online. As the web browser is the main interaction tool when working with \ac{SaaS} products, the user interfaces are created web-based. \cite{hill_guide_2013}\\
\citeauthor{hill_guide_2013} summarizes the characteristics of \ac{SaaS} software as follows:
\begin{itemize}
    \item Software that is available worldwide via the Internet free of charge or for a fee
    \item Collaborative
    \item Automatic update of the software product by the manufacturer
    \item All users use the same version of the application
    \item Software product automatically scales as needed
    \item Distribution and maintenance costs are reduced due to central management in the cloud
\end{itemize}
The popularity of \ac{SaaS} products is steadily increasing. Not only because it is easy for the end user to use such software without having to install anything on his computer. But \ac{SaaS} products are also becoming more and more interesting for smaller companies. Often, payment plans are based on the number of users of the products. Therefore, even small businesses can afford software, which was not possible before. \cite{sury_software-as--service-modell_2020}\\
In the last five years alone, the number of \ac{SaaS} products used in companies has increased tenfold. \cite{stastista_saas_2021} At this high rate, the need for standardization to align software is pressing. Naturally, when thinking about alignment, a software architect thinks about connecting systems via application interfaces. These interfaces are often well documented and offer a variety of options. But what about maybe the most important interface of a software, the user interface?
\subsection{User intefaces in \ac{SaaS} products}
Users are still the main consumers of \ac{SaaS} products. And that will not change in the next ten years. More and more products will emerge and one will be replaced by the other. Depending on what works best for the current use case. The products used in businesses could change from month to month. For users, this means they will have to learn new structures and the workflow of each software from scratch over and over again. Of course, there may be tutorials, training and documentation. But in real life, users learn how to use the software by using it. \\
A good example of this is a blog post by \citeauthor{sernoff_website_2021} that shows how many different implementations of a navigation menu exist. No wonder the sheer amount of variants can confuse users. Remembering every interaction pattern for every piece of software used can take some time. And this example refers only to navigation menus. Usually, user interfaces consist of many different components, which leads to even more confusion when switching between different programs. \\
To solve this problem, the software programs must have similar user interfaces. For comparison: The arrangement of the pedals in motorized vehicles is always the same. The gas pedal, brake and clutch are always arranged in the same order. Thus, the user can drive any vehicle without thinking about how to operate it. Standardization once defined this arrangement, and every vehicle manufacturer must comply with it. There is also a standard for software user interfaces on how human interfaces should be designed. ISO/IEC 9241 establishes a set of rules for the ergonomic design of human interaction with computer systems. Unlike the automotive industry, there are no rules for the design of user interfaces for software products. Of course, there are exceptions, such as government websites, which must be accessible and meet certain requirements. But no one can force software manufacturers to follow a standard implementation. \\

So the goal is to get software vendors to implement some kind of standardization. Providing another standard that is a little better tailored to the use case of \ac{SaaS} products may not be enough. Large standards like ISO/IEC 9241 are often far too complex to understand and don't provide pre-built components to start with. An optimal solution for implementing standardization should be easy to use for those doing the work, which are the developers. \\
In the world of user interface development on the web, many are familiar with frameworks, templates, or guidelines. They are simple and widely used in the developer community. But is it possible to have something that provides standardization for user interfaces? Such a assistant must be easy for the developer to use. In the best case, the developer doesn't need to know anything about patterns or rule sets when using it. \\
Another question that arises with standardization: Does it even make sense to have a standard for design and structure of user interfaces of \ac{SaaS} products? Doesn't that lead to a world where every product looks the same? Products want to stand out from one another, on the one hand, through their capabilities, but also through their style and handling. Therefore, such an assistant must be customizable to the needs of the company. The freedom of design cannot be restricted by such a tool. The components must have the ability to adapt to the product design. \\
Assuming a tool meets all the requirements just described, i.e., it is easy to use for development and is customizable enough to stand out from other products. One final question remains: Is a user who works with a product that provides this support more productive? He should be, otherwise it makes no sense to use such a tool. \\

In examining how \ac{SaaS} product companies are trying to align their user interfaces across their entire product portfolio, a new trend is emerging - design systems. So why not use a design system to bring not just one product line, but all \ac{SaaS} products into alignment? The \citetitle{uswds_uswds_nodate}, for example, is already trying to unify all government-related software into one design system. So it should be possible to go a step further and standardize an entire type of application with one design system. \\
Many large \ac{SaaS} companies have already developed their own design system. The \citetitle{limcaco_design_nodate} website summarizes them in a list. With 104 design systems from various \ac{SaaS} vendors, this seems to be a hot topic with a lot of potential. Researching these design systems, gathering best practices, and building a base design system for all companies could lead to a new kind of standardization through a design system. \\

This makes it clear how this elaboration will approach the standardization of user interfaces. With the help of design systems. The next chapter first introduces the general topic of design systems. This will be necessary later to understand how the designed design system got its architecture.
%\subsection{Vision of standardized \ac{SaaS} products} 
\section{Fundamentals}
\input{chapter_2/user_experience}
Design systems typically support web-based products to ensure a consistent user experience across all products. Software solutions are moving from on-premises to the cloud. The web browser is already the new user interface for most users. Design systems help keep a company's products aligned. With the help of a central system that provides not only components, but also guidelines and patterns.  \citep{macdonald_practical_2019} \\

Design systems are most similar to component libraries and style guides. This comparison occurs because a design system consists of the same elemental building blocks as the other two. It is essential to understand that a design system includes design processes and philosophy. It builds an agreed-upon basis for discussion between product managers, designers, and front-end developers. \cite{vesselov_building_2019} \\
The literature defines two definition for design systems:
\begin{tcolorbox}[title=Definition of design system by \citet*{macdonald_practical_2019}]
A design system is a single source of truth for shared parts and processes, such as components, patterns, and guidelines, to build consistent products. [...] Additionally, design systems reflect the culture, team values, and visual language of an organization.
\end{tcolorbox}
Another definition goes even further and specifies the aspect of documentation of design systems:
\begin{tcolorbox}[title=Definition of design system by \citet*{vesselov_building_2019}]
A series of documented elements, components, and patterns that include both design and front-end guidelines. The documentation contains live code examples, allowing cross-functional teams to easily reuse styles and components in several instances across an application. A design system also includes underlying design principles, rules, and guidelines that help a team build one or multiple products.
\end{tcolorbox}
A basic structure of design systems can be derived from both definitions. According to this, a design system can be divided into three different parts.\\

On the one hand, there are the guidelines, which provide users with instructions on how to use the design system and build how to build the software product. As a further subdivision are the components, which a developers or designers use. These must be made available as simple as possible so that they can be easily integrated into the end product. The last and most underestimated part is the documentation mentioned by \citet*{vesselov_building_2019}. The best components can be provided, but without clear and interactive documentation, they are only half as good. \\

These sections will be discussed further in the remainder of this chapter. \cite{macdonald_practical_2019}\cite{vesselov_building_2019}

\subsection{Guidelines}
Guidelines are an important part of a design system to differentiate from a component library. Of course, component libraries also have guidelines to ensure that developers use them correctly, but these are technical in nature.  \\
Guidelines in design systems are to serve as communication assistance between designers and all other involved ones. In the literature, guidelines are also referred as a common design language for a company.\\
The manual effort of handoffs between UX and the development team can be handled more efficiently. Many issues can be addressed up front with well thought out guidelines, allowing developers and designers to work more efficiently. In case there are still open design questions, the guidelines from the design system serve as a basis for discussion. \cite{vesselov_building_2019} \\

It is important to mention that guidelines are not static images and long texts. They must automatically grow with the design system and be well structured. At best, a design system is structured so that the listed components document themselves. \\
A good structure also includes a user-friendly navigation, which allows to find and search for components or documentation of any kind in the design system. This is supported for example by autocomplete searches, overview pages and tables of contents.  \cite{macdonald_practical_2019}\cite{vesselov_building_2019} \\

As in software development in general, it is common to apply versioning in design patterns. Additionally to the code, versioning of the documentation and guidelines becomes crucial, especially within large projects. This extra effort, allows the entire team to track policy changes and implement them correctly for the version used by the developer. \\
Issue trackers are common for large projects. Therefore, it makes perfect sense to use one in design systems as well. It allows users to report bugs and improvements. \\
Furthermore, release notes with a detailed description by image and text help motivate users to follow the design system and stay up to date. This behavior makes updating to new versions convenient for the developers. \cite{macdonald_practical_2019} \\

But it's not just developers who benefit from the guidelines. For example, a product manager may have an idea for a new feature that he wants to present to customers. However, the UX team doesn't have the resources to mock something on the fly. The guidelines still allow the product manager to create a proof-of-concept within the given capabilities and present it to the customer. Properly applied, the product manager can be confident of meeting UX requirements.  \cite{vesselov_building_2019} \\

Another use case for design system guidelines is companies' onboarding processes. The developer, as mentioned above, and product managers can use the guidelines to find their way around the product faster. Still, people from sales or marketing can also use this resource. \cite{vesselov_building_2019} \\

%In the literature, these guidelines are also referred to as a common design language for a company. \\
\citet*{vesselov_building_2019} divide guidelines into four different types.

\subsubsection{Formal definition} 
Description of a component, and function. For the developer, the documentation may seem trivial, but it helps avoid misunderstandings. \cite{vesselov_building_2019}  \\
Figure \ref{fiori_action_list} shows that a simple explanation of trivial components is sufficient to prevent misunderstandings. In addition, the visual representation of the elements helps complete the guideline Usage guidelines.
\begin{figure}[ht]
\centerline{\includegraphics[width=\linewidth]{images/fiori_action-list_formal.png}}
\caption{SAP Fiori Action List formal guideline \cite{sap_fiori_nodate}}
\label{fiori_action_list}
\end{figure}
\newpage

\subsubsection{Usage guidelines} Usage guidelines help to understand how to use components. In addition, these guidelines explain how the corresponding parameters of a component work. This way, the user knows everything he needs to use the component in the product. \cite{vesselov_building_2019} \\
As can be seen in Figure \ref{atlassian_button}, the guideline provides guidance on where the component should be used. Sometimes usage guidelines also offer anti-patterns to the users. Anti-patterns tell users of the design system where they should not use that component.
\begin{figure}[htb]
\centerline{\includegraphics[width=\linewidth]{images/atlassian_button_usage.png}}
\caption{Atlassian Design System Button usage guideline \cite{atlassian_design_system_atlassian_nodate}}
\label{atlassian_button}
\end{figure}

\subsubsection{Technical guidelines} \label{tech_guideline}
\begin{wrapfigure}{r}{7cm}
	\includegraphics[width=7cm]{images/pluralsight_date-picker_technical.png}
	\caption{Pluralsight Design System Date picker technical guidelines \cite{pluralsight_ds_nodate}}
	\label{pluralsight_date_picker}
	\end{wrapfigure}
% Left here
This part of the guide is mainly intended for developers who are to implement the documented component in the software. Technical guidelines clarify ambiguities in implementation.  \cite{macdonald_practical_2019} \\
The same way as the corresponding parameters configure the appearance or behavior of a component. An often-found feature here is copying code snippets. This allows developers to immediately copy and paste the component code into their code. \cite{vesselov_building_2019} \\
Figure \ref{pluralsight_date_picker} shows a good example.The implementation of Pluralsight's Date Picker is explained very well here with a code example. The code snippets can be copied directly, opened, and tried out in a code sandbox.

\subsubsection{Related components} Linking components and patterns help the user explore the design system. Also, this can support the creative process for designers and developers. \\
Figure \ref{servicenow_accordion} is an example of a linked component in a design system. Such a representation of a linked component not only helps with creativity, but also lets developers and designers move more quickly through the collection of components. \\
This way, if there are problems with the implementation, the developer can quickly understand the relationship between the used components and resolve the issue. \cite{vesselov_building_2019}
\begin{figure}[htbp]
\centerline{\includegraphics[height=150px]{images/servicenow_accordion_related.png}}
\caption{Servicenow Accordion related components \cite{servicenow_servicenow_nodate}}
\label{servicenow_accordion}
\end{figure}

\subsection{Design Principles}\label{design_principles}
Design principles should be a central consideration when building a design system. At the same time, the design principles should reflect the norms and values of the product organization. In doing so, the points established do not follow any rules, except that they are set collectively by the team. \\
Thus, these design principles serve as a basis for design discussion and decision-making. By simply self-explanatory principles, the designers and developers can quickly create new designs without a significant coordination effort. Often questions are taken as principles. This gives developers and designers the impulse to ask themselves whether they follow the design principles during the creation process. \cite{brignell_design_2022} \\
As an example, \citet{berners-lee_principles_2013} has developed design principles for the web: 
\begin{itemize}
\item \textbf{Simplicity} - Simple solutions are better solutions
\item \textbf{Modular Design} - Change things and it will only affect one part
\item \textbf{Being part of a Modular Design} - Realize you own the Design System
\item \textbf{Tolerance} - "Be liberal in what you require but conservative in what you do"
\item \textbf{Decentralization} - Don't produce bottlenecks, allow scaling in any direction
\item \textbf{Test of Independent Invention} - "Designing a system not to be modular in itself, but to be a part of an as-yet unspecified larger system."
\item \textbf{Principle of Least Power} - Use matching tools for the matching tasks
\end{itemize}
Design principles can be quite different. In this case, they are relatively technical, as the team wants to focus on this topic. Other organizations, such as Adobe with Adobe Spectrum, have people at the forefront. \cite{spectrum_adobe_spectrum_nodate} \\

Many underestimate the importance of these anchor points. Design principles play a central role in a design system.  The product team should not only build software based on these principles, but also gain an understanding of the big picture of the product organization. Therefore, the companies' design strategy is in line with their design principles. In this way, not only is the existing product organization aligned with the design principles, but new employees can also use them to integrate themselves more quickly into the product organization. Design principles are effective when they serve as a guide for the creative process.\cite{vesselov_building_2019} \\

First and foremost, a team must create design principles. A single person should not establish design principles on his or her own, even if it is possible to do so. Developing design principles as a large group reflects the creative thinking of everyone and helps align the entire company with these values. \\
Here, \citet{vesselov_building_2019} lay out a 5-step plan to iteratively create and continuously improve them. Figure 5 summarizes these  \ref{design_principles_steps} steps:
\newpage


\begin{figure}[htbp]
\centerline{\includegraphics[width=\linewidth]{images/design_principles_steps.png}}
\caption{5 steps to introduce design principles inspired by  \citet{vesselov_building_2019}}
\label{design_principles_steps}
\end{figure}
Following these steps, a team creates design principles and implements an iterative process to update the design principles on a regular basis. \\
Next, the responsible team shares the values of the created design principles with the rest of the product organization. In this step, many design systems create so-called design blogs or design news. \\
They not only help the organization keep track of the design principles in their daily work. They also provide a platform to share updates with users. Moreover, there is an exchange on related topics on these platforms. This opens communication, in turn, helps to improve and keep the design system and principles up to date.  \cite{google_material_2022} \\

This chapter lays the foundation with the guidelines and design principles, while the next chapter will introduce another building block: the component library.
\subsection{Component Library}
The technical counterpart to the description guidelines and design principles is the component library. Both cannot exist in a design system without each other.  By definition, a component is "a constituent part"  of in this case a user interface. \cite{component_definition} Combined with library, which means a "collection of something", it results in a collection of important constituent parts for user interfaces. \cite{library_definition} \\

The definitions existing from the literature are as follows:
\begin{tcolorbox}[title=Definition of component library by \citet*{vesselov_building_2019}]
A set of styles and components that can be used and shared among a team. A component library consists of common core elements that are used throughout an application. [...] Component libraries may or may not include living code. [...] Unlike UI frameworks such as Bootstrap, component libraries are tailored to specific purposes, like an internal brand.
\end{tcolorbox}
In addition to components, there are also styling rules that also include layout specifications in component libraries. 
\begin{tcolorbox}[title=Definition of component library by \citet*{macdonald_practical_2019}]
Component libraries, UI libraries, or code libraries provide frontend code for UI components (a.k.a. widgets, modules, chunks, blocks). Internally, you might use a component library as a shared collection of UI snippets implementing patterns that anyone in the organization can contribute to building.
\end{tcolorbox}
An important point that emerges from these definitions is that component libraries have a clear focus on internal use. This is also the main difference to UI frameworks. \\

After reviewing various design systems, component libraries are divided into 4 categories:
\begin{itemize}
	\item \textbf{Layout} - Spacing and presentation of content placement on a site
	\item \textbf{Styles} - Color definitions, Typography, Icons
	\item \textbf{Components} - Reusable parts fulfilling one purpose
	\item \textbf{Patterns} - Combination of multiple components 
\end{itemize}

\subsubsection{Layout}
The foundation of any design system is the ability to place, move, or arrange elements on web pages. To achieve a consistent appearance in applications, it is important to align spacing and positioning in a central place. \\ 
For this purpose, so-called design tokens in form of CSS variables are often used. Similar to Tailwind (\url{https://tailwindcss.com/}), CSS classes implement design tokens and abstract them, so the developer does not need to use the variables natively. Thus, the developer does not need to know anything about CSS.  \\
Salesforce Lightning Design System, for example, lists all layout tools under \textbf{Utilities}. It sets sizes for text, simple boxes, and spacing between components. In fact, a sophisticated grid layout system can be seen as well. \\
\begin{figure}[hbtp]
	\centerline{\includegraphics[width=\linewidth]{images/salesforce_lightning_layout.png}}
	\caption{Salesforce Lightning Design System grid layout \cite{lightning_design_system_lightning_nodate}}
	\label{salesforce_lightning_layout}
\end{figure} 

Figure \ref{salesforce_lightning_layout} shows that CSS classes achieve the desired layout. Furthermore, both dynamic and static layouts are supported. So the user has all possibilities to use the layout of the design system.\\

Besides the usual regulations for spacing and positioning, other points like visibility, scrollability or printability can also be described in the layout. Basically, all requirements for visible elements in the target system can be specified by the default layouting in the design system.

\subsubsection{Styles}
Styles are about colors, typography and icons. These topics may seem trivial, but it is important to make a connection to the chosen design principles. This helps to convey that design language in the other disciplines of the component library. \\
A style system provides enough freedom for design decisions so that the implemented products still have the possibility to have a unique look and feel. Some component libraries ensure this by allowing customization of basic style parameters.\cite{vesselov_building_2019}

\paragraph{Color}
A good first step is the introduction of a color system to build a design system. Colors are important for the aesthetics of a product.
Without changing the functionality or layout, the product team can change the colors until they fit the product. \\
\begin{figure}[htbp]
	\centerline{\includegraphics[height=8cm]{images/adobe_spectrum_color_palette.png}}
	\caption{Adobe Spectrum color palette \cite{spectrum_adobe_spectrum_nodate}}
	\label{adobe_spectrum_colors}
\end{figure}
It is always important to keep accessibility in mind when choosing colors. For a specific text size, it is important to maintain a certain contrast on colors. The WCAG (\url{https://www.w3.org/WAI/standards-guidelines/wcag/}) describes this contrast with two levels AA and AAA. This contrast is especially important if a product has several long paragraphs.  \\

Adobe's Spectrum Design System in Figure \ref{adobe_spectrum_colors} illustrates that design systems provide not just one color, but an entire color palette. The colors are often divided into primary, secondary, text color, background color, accent colors, shadows and so on. \\
Again, CSS variables with the appropriate names consume color tokens to provide them later for product implementation. The challenge is to offer a well-defined color palette, but at the same time not to overload the user. Too many colors quickly confuse users about which color to use for the feature. \cite{vesselov_building_2019}

\paragraph{Typography}
A typography system has two different categories. \\
First, there is typography for smaller text elements. Such elements consist of a maximum of three words. Examples are headings, buttons or labels. \\
Second, typography determines the appearance of long paragraphs. Paragraphs can have a different weight, size, or even a different font family altogether.
\begin{figure}[hbtp]
	\centerline{\includegraphics[height=5cm]{images/gov_uk_typo.png}}
	\caption{GOV.UK Design System caption typography example \cite{govuk_govuk_nodate}}
	\label{gov_uk_typo}
\end{figure} \\
In both cases, however, the general typography must follow the defined layout system. Texts rely on defined design features for spacing and padding. Figure \ref{gov_uk_typo} shows a helpful tool, an overview of all typography options supports the selection of the right design.  \cite{vesselov_building_2019}


\paragraph{Iconography}
The Icongraphy is a bit of a specialty in a design system. It may not seem to be important for a product but icons often deliver a lot of the overall design language to the end user. \\
Although icons are important to design language, it is not always necessary to create new iconography. A good start is to use an iconography that is already widely used. Later, a custom iconography offers a lot of potential to visualize the company's design language in it. \\
As with other parts of a design system, documentation is key. Focus on how to create new icons, what proportions and shapes. If done right, it's an easy step to introduce new icons into the iconography. 
\\
To make them accessible to the team, an iconography describes how to add them to the product. As in the technical guidelines in \ref{tech_guideline}, a code snippet explains their use to the developer. \cite{vesselov_building_2019}

\subsubsection{Components}
Components are the heart of a component library. Moreover the building blocks for applications. They are built with all the tools and fundamentals just presented.Components are highly reusable and as flexible.  \\
\begin{figure}[hbtp]
	\centerline{\includegraphics[height=5cm]{images/zendesk_component_example.png}}
	\caption{Zendesk Garden dropdown component \cite{zendesk_garden_zendesk_nodate}}
	\label{zen_garden_component}
\end{figure}

Combinations of layout and style specifications result in components for different use cases. Also, Figure \ref{zen_garden_component} shows that a component that is widely used in different design systems is a dropdown component. Because native dropdowns are often unable to meet full user experience requirements, component libraries have often replicated this component.\\
By using these predefined components in applications, the developer ensures that the guidelines are adhered to. For certain combinations of components in an application, there is another term called patterns, which is explained in more detail in chapter \ref{patterns}. \\

\begin{figure}[hbtp]
	\centerline{\includegraphics[height=5cm]{images/zendesk_component_interface.png}}
	\caption{Zendesk Garden dropdown interface \cite{zendesk_garden_zendesk_nodate}}
	\label{zen_garden_interface}
\end{figure}
To achieve flexibility and reusability, it is important to always have the component's interfaces in mind. Well-defined interfaces are the key to successful components. A possible solution, as in Figure \ref{zen_garden_interface}, is good interface documentation. A simple table with property values, the expected type and a short description is sufficient. In some design systems, the developer can change these input properties in a live demo so that the developer can see the changes immediately.\\

As with other parts of a design system, documentation is key. To get well-documented interfaces, always involve the engineers who are actually building the product to receive their feedback.  A well-structured and systematic approach to how a component works helps developers use the components.  \cite{vesselov_building_2019}\\

The above-mentioned components alone are not enough. Sometimes it takes more than one component to get the job done. The next chapter explains how patterns achieve documentation of a complex component construct. 

\subsubsection{Patterns} \label{patterns}
Often, a combination of certain components is used over and over again in different places and even in different applications. To cover this use case, a design system has a concept called patterns.\\ 
Patterns specify a combination of several components and document the composition of these components. Patterns are an extended documentation to reproduce a certain combination of components in an application. \\

\begin{figure}[htbp]
	\centerline{\includegraphics[height=7.5cm]{images/atlassian_abstract_form.png}}
	\caption{Atalssian Design System abstract form layout \cite{atlassian_design_system_atlassian_nodate}}
	\label{atlassian_form_layout}
\end{figure}
Atlassian's design system provides an excellent example of a pattern with its forms. Figure \ref{atlassian_form_layout} shows an abstract visualisation of how input fields and buttons should be placed when a form is on a single page. This abstraction leaves no room for interpretation, and developers and designers can continue with their work. \\

Patterns refer not only to the interaction of components, but also to the switching of visualization due to changes in a user's permissions, for example. A pattern documents all UI changes related to the user experience, no matter how small. \cite{vesselov_building_2019} \\

Most often, patterns appear in the general navigation of applications. Such navigation patterns evolve naturally from introducing them into products and grow iteratively with more and more feedback from the product development team.
\\
A good example for a pattern is the navigation bar. By combining components such as buttons, dropdowns, and images, a navigation bar is created. Together with skeleton code around these components documented in a pattern, they form a complete navigation area within an application. 

 
\section{Implementation}
\subsection{Architecture}
Modelling the architecture of a system that supports customization of SaaS product user interfaces is not a trivial task. Often, user interfaces for products are developed only to support their purpose. Finding common ground for multiple products can be difficult.  \\
Design systems define a common place where the company's products can align. Why is it not possible to develop a central system to create a common idea for user-friendly SaaS products? Not only do well-designed components help align, but a design system foundation with guidelines and principles helps developers and designers create a good product. \\
Ten different design systems from well-known SaaS products as well as design systems with a common purpose help to model a new design system. The goal is to identify best practices from the SaaS world and understand the needs of companies using design systems.
\begin{table}[!ht]
\begin{tabular}{|p{0.2\linewidth} | p{0.7\linewidth}|}
\hline
 \textbf{Name} & \textbf{Description} \\ \hline
ServiceNow \cite{servicenow_servicenow_nodate}  & Platform design system.  Guidance to create components and upload them to the platform. \\ \hline
Adobe Spectrum  \cite{spectrum_adobe_spectrum_nodate} & User centralised design system. Many well designed components with matching guide to deliver a great experience. Built in web components and react components. \\ \hline
Zendesk Garden \cite{zendesk_garden_zendesk_nodate} & Basic design system with guidelines, components and patterns. Tailwindcss integration. Built in react components. \\ \hline
Atlassian Design System \cite{atlassian_design_system_atlassian_nodate} & Design System connected with company values. A lot of guides on how to use designs, components and to write content. Includes also employee motivation. Built in react components. \\ \hline
Base Web  \cite{base_base_nodate} & Open source design system. Used by Uber. Providing a blog and guides on how to use the base design system. Design System intended to be used as baseline and should be overwritten when used. No principles or values included. \\ \hline
SAP Fiori  \cite{sap_fiori_nodate} & Standard design system. Focus on accessibility and multiple device support. Including many themes for different applications. Delivers a toolkit to better use the design system as a designer.  \\ \hline
GOV UK Design System  \cite{govuk_govuk_nodate} & Not really a design system. Missing guidelines and principles. Externals can propose changes. Providing CSS classes for HTML elements.  \\ \hline
Lightning Design System \cite{lightning_design_system_lightning_nodate} & Design System to support developers and designer at their work. 4 principles with a clear message to align every user. Guidelines and best practices on many topics.  Components are built with pure CSS classes. \\ \hline
Google Matrial Design \cite{google_material_2022} & Open source design system. Providing the user with design principles which helps to understand the usage of the design system. Material Design provides only components and no patterns. Blogs and further resources are helping additionally to the guidelines. Components are built with pure CSS classes. \\ \hline
Pluralslight Design System \cite{pluralsight_ds_nodate} & Design System without principles and guidelines. For the moment only components are present. Providing a workflow for developers and designers to contribute to the design system.  Only few patterns. Built with react components.  \\ \hline
\end{tabular}
\caption{\label{tab:design_systems_in_the_wild} Overview of 10 existing design systems}
\end{table}
As can be seen in table \ref{tab:design_systems_in_the_wild} for these ten examples of design system, the interpretation of one can vary. \\
A good reference for a design system with a suitable use case is the Base Web Design System. Its purpose of providing a base of styles and functional components helps developers customise for their products. Also, the fact that this design system is open source underlines that this design system has been developed by the community and is not controlled by a corporate design team. Built-in accessibility is also a requirement that must be present in a common design system. The instructions on how to extend and use the design system are also a perfect reference. \\
But there are also disadvantages. The Base Web Design System lacks guidelines and principles that are crucial for a design system. A look at Google's Material Design shows that even an open-source and versatile design system can have design principles. Design principles help developers get an idea of how to develop and design with the system. Therefore, design principles are indispensable in a design system and should not be missing. \\ \newpage
Another reason, is the use of React for creating the components in the Base Web Design System. As a design system that should be used by everyone as a standard for the implementation of SaaS products, it is therefore unsuitable. As developers are expected to use React as a frontend library. Looking at other examples such as the GOV UK Design System or Salesforce's Lightning Design System shows that it is possible to create components using web standards that can be used by anyone without having to use a library. \\
With these requirements, an architecture can be drawn as can be seen in figure \ref{architecture_sds}. \\
\begin{figure}[htbp]
\centerline{\includegraphics[width=\linewidth]{images/architecture_sds.png}}
\caption{Architecture of SaaS Design System}
\label{architecture_sds}
\end{figure}
According to the description of a design system presented in chapter 2, the SaaS design system can be roughly divided into four parts.
\subsubsection*{Design Principles}
In terms of design principles, the \ac{SDS} strives to keep them lean and easy to understand, as the design system is intended to serve as a basis for other design systems. The \textbf{Simple} principle states that component design should not have unnecessary styles or features that make it difficult to extend. \\
Many products strive to implement accessibility in their products. With the \textbf{Accessible} principle, \ac{SDS} emphasises the importance of accessible user interfaces. This is not only important in terms of inclusion, but also helps accessibility in the overall user experience for all users. \\
The third and final principle is \textbf{Solid}. As stated earlier, the \ac{SDS} should be a foundation for other design systems to build upon. Therefore, the importance of a stable and consistent API is very important. For this reason, the design system has deliberately chosen \textbf{Solid} as the third and final principle. \\
\subsubsection*{Guidelines}
The \ac{SDS} guidelines are based on the design principles just presented. In addition to the core guidelines on extension, accessibility and basic use, there are also guidelines that focus on contributions and collaboration. As this is an open source system, as many people as possible should be able to work on it. \\
The extension guidelines address how to integrate \ac{SDS} as the basis for a company's own design system. It shows developers and designers how to create their own from the components provided. \\
Since accessibility is also a design principle, there must be a guideline that defines what the \ac{SDS} means by accessibility. It should give the user a definition and sources for accessibility. But also self-designed components should have a guideline that helps users to implement accessibility. After this guideline, the user should have no more questions about accessibility. \\
As a third guideline, the \ac{SDS} will support the user in using the system. This guide could also be seen as an entry guide and will cover the basics. Importing the design system, proper bootstrapping and guidance on configuring the system. This may seem self-explanatory, but the lack of these guidelines often prevents users from using the system effectively. The user guide should be as simple as possible and cover every small step needed to get started with the \ac{SDS}. \\
One goal of this design system is to be developed by the community for the community. However, this can quickly get out of hand if everyone contributes without any guidance. Therefore, it is important to introduce some from the beginning. This guide provides guidance on how to contribute to the component library, but also on how to enforce changes to the guidelines and principles. As this design system is not set in stone, there should be opportunities to change and adapt everything. What this will look like in the end will evolve over time. Some ideas could be a voting process or an RFC (source) process, as is common in the software industry. \\
To achieve high interactivity in such a design system, some design systems introduce blogs and forums for knowledge exchange. In this way, users can connect, discuss and contribute to ideas to further improve the design system. The most important point is the moderation of such an interaction platform. A well-moderated blog and forum will further enhance the community around the \ac{SDS}. \\
Overall, it can be said that the guidelines for the \ac{SDS} are aimed at building a community around the design system to contribute and enable users to create a community design system. 
\subsubsection*{Components} \label{sds-component}
Without well-assembled components, design systems cannot exist. Therefore, choosing the right technology package for building components is very important. 
In the case of \ac{SDS}, one of the most important requirements is that the system is usable independently of the front-end framework used. To achieve this, \ac{SDS} uses the web components that are supported in almost all modern browsers. As it is possible to create web components without importing libraries or frameworks, it is for \ac{SDS}. \citep{mdn_web_component_nodate} \\
Creating web components natively can be complicated. For this reason, the Lit framework was designed to make it easier for developers to create web components. With a focus on ease of understanding, intelligent DOM updates and small package size (5 KB), Lit is a perfect complement for creating components for design systems based on standard web components. \citep{lit_nodate} \\
To further assist developers, \ac{SDS} uses Typescript, a superset of Javascript. It extends Javascript with types and interfaces. Typescript must be compiled into Javascript for the browser to understand it, but this allows the developer to find errors much faster because it fails at compile time rather than at runtime. \citep{microsoft_typescript_nodate} \\
To use and provide design tokens for colours or spacing, the \ac{SDS} uses custom CSS variables. They are imported into the root element during bootstrapping of the design system. The definition of the design tokens can be found in the documentation. \citep{mdn_css_vars_nodate} \\
Last but not least, the design system components created must be documented and accessible to users. To start \ac{SDS}, Storybook is used, which has a lot of built-in functions that support documenting the components. With MDX, the combination of Markdown templating (MD) and code injection via JSX, it is possible to write fluid documentation without having to jump back and forth between files. \citep{otander_markdown_2017} \\
With this technology stack, \ac{SDS} provides users with highly reusable web components that are not only easy for users to access, but also easy for contributors to develop. 
\subsubsection*{Patterns}
An essential part of the \ac{SDS} are the patterns. They describe how user interfaces should be designed to align with the core capability of this design system. The patterns help developers automatically apply best practices and web standards without having to read an entire text. \\
Patterns are created using components and standard HTML elements provided by the design system. These are easily accessible to the developer by copy and paste. With additional documentation on when to use them and when not to use them. \\
The patterns are supported by additional guidelines on how they can be adapted and redesigned. This enables developers to meet the requirements needed for their own design system.\\
Accordingly, live examples show the developer how the patterns will work in the final product. With several different examples for each pattern, the possibilities for customising each pattern are presented. \ac{SDS} users also have the opportunity to share their creations and application of patterns below the documentation. 

\subsection{\ac{SDS} components}\label{sds_button}
An essential part of the implementation of a design system is the components. As described in chapter \ref{sds-component}, the design system builds its components using the Lit framework. This chapter presents an example of the button component of the \ac{SDS}. In addition, it shows a complete example of a component constructed with design tokens and documented with the Storybook.  \\
The components use TypeScript to take advantage of custom decorators. The decorators provided by Lit further simplify the boilerplate code for creating a web component. \\
\lstinputlisting[linerange={1-4},firstnumber=1,caption={Initialization of \ac{SDS} button component},label=ButtonInit]{../Code/src/components/button.component.ts}
In listing \ref{ButtonInit}, the \texttt{@customElement} decorator initializes the component by passing the tag name as a string and appending it to an \ac{ES6} class. For it to work correctly, the class must extend the Lit Element class. Finally, the web component registers itself, and users can use the component with the defined tag, in this case \texttt{<saas-button>}. \\
In order to see something when the web component just created is used, it must implement the render method. This method expects a \texttt{TemplateResult}. 
\lstinputlisting[linerange={19-26},firstnumber=19,caption={Rendering of \ac{SDS} button component},label=ButtonRender]{../Code/src/components/button.component.ts}
Lit element provides the import of an \texttt{html} string literal that constructs the \texttt{TemplateResult} expected by the render function. With this functionality, Lit provides an efficient way to respond to variable changes and intelligently update the \ac{DOM}. \\
Properties also use custom decorators and declare properties on web components by writing \texttt{@property} in front of a class attribute (line 19-20, Listing \ref{ButtonRender}). Thus, Lit Element implements change detection and add automatic type conversion. For a detailed description of the capabilities of this decorator, see the documentation. The property value of the created button web component changes and the constructed template string reacts to these changes and updates the displayed template accordingly. \\
Last but not least, the web component must use the defined design tokens. Since \ac{CSS} variables define these tokens, components consume them as follows: 
\lstinputlisting[linerange={5-20},firstnumber=5,caption={Styles of \ac{SDS} button component},label=ButtonStyles]{../Code/src/components/button.component.ts}
The Lit framework implements a static property of the \texttt{styles} of its classes. The component uses a literal \texttt{css} string to convert a string into the required \texttt{CSSResult} type. This way, it would be possible to consume design tokens via input properties. However, to manipulate design tokens in one place, the components of \ac{SDS} will use \ac{CSS} variables. It is possible to customize specific tokens throughout the design system by adapting them. Each component in the \ac{SDS} will use the defined tokens. Line 7-9 in listing \ref{ButtonStyles} shows an example of design tokens for the button component in the \ac{SDS}. \\
Finally, the button component is displayed in Storybook, the documentation tool for \ac{SDS}. Due to the limited time frame of this elaboration, the documentation is kept short. Figure \ref{storybook_button} is an example of the documentation of \ac{SDS} components. As in other mature design systems, the subsequent development of \ac{SDS} will add a full version of the documentation. \\
\begin{figure}[htbp]
    \centerline{\includegraphics[width=\linewidth]{images/storybook_button.png}}
    \caption{Example documentation of the \ac{SDS} button component}
    \label{storybook_button}
\end{figure}
At the moment, the documentation consists of a short description that briefly explains to the developer how and where he can use the button in his applications. In addition, a live example of the component is presented, automatically linked to the properties shown below. When the properties are changed, the example above is updated. This instant feedback gives developers or designers who want to use this component the chance to determine which configuration is most suitable for their use cases. \\
Inspecting the \ac{DOM} element of the button component, the element explorer looks like Figure \ref{button_element_explorer}. The new web component, the \texttt{saas-button}, is a valid \ac{DOM} element. A new shadow root is opened inside the button component. For this, the button component opens a new shadow root. The shadow root allows the web component to isolate styles and elements from the global document. The button component defines the elements used to display in this shadow root. Also, the code defines the styles for the button at the shadow \ac{DOM} level, so the rest of the \ac{DOM} tree never receives these styles. This technique helps to keep elements and styles separated and easy to understand. \\
\begin{figure}[htbp]
    \centerline{\includegraphics[height=100px]{images/button_element_explorer.png}}
    \caption{Inspection of \ac{SDS} button in element explorer}
    \label{button_element_explorer}
\end{figure}
It is crucial to declare that \ac{CSS} variables in the document's root are also available in the shadow \ac{DOM}. In contrast, style declarations, e.g., for the button element in the overall document, are not applied to \ac{DOM} elements in the shadow \ac{DOM}. \\
The \texttt{<slot>} element in the shadow \ac{DOM} is a default placeholder for anything inserted into the \texttt{<saas-button>} element when using the web component. The content is projected within the shadow \ac{DOM} and inserted in the place of the \texttt{<slot>} element. Slots make it possible to create nested elements with web components which is helpful in many different use cases.  \\

The example of a simple button component shows how to create and use the components of \ac{SDS}. From here, building all the different components needed to create patterns is trivial. The building is simple because patterns, as described, are a combination of components that applications reuse. The missing piece to using the \ac{SDS} in an application is integration, which the next chapter explains.
\subsection{Design System Integration}
The last step to complete the user story of the SDS is to integrate the system into an application. This chapter explains how to package the SDS and load it into a desired application. It is important that the integration is simple so as not to discourage developers from using the design system. \\

The build process is the first part that is important to understand the integration workflow. Webpack is the build tool used by SDS. As shown in Figure 3 in Chapter 1, the components of the design system are built using Typescript, including the Lit framework and using SCSS for styling. Since Typescript and SCSS are not supported by the browser, they must be processed beforehand. Therefore, Webpack comes into play to compile the code. \\
Webpack uses a configuration file, often called webpack.config.js, to describe the steps by which the code is compiled. This file describes rules that tell Webpack what to do with which files that come into the build pipeline. These rules for SDS look like this:\\
\lstinputlisting[linerange={23-41},firstnumber=23,caption={SDS Webpack rules},label=WebpackRules]{../Code/webpack.config.js}
The rules define regular expressions that are used to assign files with their extensions to the corresponding compilers, which are called loaders in the Webpack world. The loaders used here are build-in. But there are also custom ones that can be integrated into a build process. When using loaders, it is sufficient to include the loader string in the use property of a rule object. For example, in line 25-26, Listing \ref{WebpackRules}, the Typescript loader is matched with the regular expression Typescript to process Typescript files. The same pattern is used in line 34-35, Listing \ref{WebpackRules} to match SCSS files with the default style loader that comes with Webpack. 
\section{User Test}
\input{chapter_1/test} 
\newpage
\section{Test results}
This chapter presents the results of the test cases performed in Chapter 4. The test was conducted in an enterprise environment. All test participants use SaaS products in their daily work. A total of 22 test runs were performed. These were divided equally between the two test systems. \\
Test candidates try to fill as many different positions as possible. This helps to give a more general meaning to the data obtained. Table \ref{tab:test_candidates} gives an overview of the selected test candidates. \\
\begin{table}[ht]
    \centering
    \begin{tabular}{|p{0.4\linewidth} | p{0.1\linewidth}|p{0.1\linewidth}|}
        \hline
        \textbf{Role} &\textbf{Normal}&\textbf{\ac{SDS}} \\ \hline
        Software Engineer & 7 & 4 \\ \hline
        Cusomer Success & 0 & 2 \\ \hline
        Intern & 1 & 0 \\ \hline
        Consultant & 0 & 1 \\ \hline
        Product Manager & 1 & 0 \\ \hline
        Customer Support & 1 & 0 \\ \hline
        Operation Manager & 0 & 1 \\ \hline
        Engineering Manager & 0 & 1 \\ \hline
        Director Engineering & 0 & 1 \\ \hline
        Customer Success Engineer & 1 & 0 \\ \hline
        IT Support Engineer & 0 & 1 \\ \hline \hline
        \textbf{Total} & 11 & 11 \\ \hline
    \end{tabular}
    \caption{\label{tab:test_candidates} Test candidates}
\end{table}

The list contains various positions from operational and lower management, but also some test candidates from higher management. Unfortunately, the majority of the test candidates are from the software development department, which could be related to the relationships of the author of this elaboration. \\

\subsection{Observation}
As a first step, this elaboration provides for the test results of the observation of the test runs. These observations were carried out entirely subjectively and provide information about the reactions of the test candidates when dealing with the two systems. \\
In every other test run, participants took a very long time to reach the home page. This shows that the distraction page works well. After the test run, participants indicated in the feedback session that they would move faster if they did not know they were being tested. Overall, participants who took a long time on the home page stopped reading after a while and immediately scrolled to the top of the page to get to the data table. There were no differences in interaction between the SDS testing system and the normal system. \\
As with the long distraction, participants show a pattern of quickly finding the "Add +" button to add a new entry in both applications. As with the long distraction, participants show the pattern of quickly finding the "Add" button in both applications to add a new entry. Without any response to the second view, participants continue adding data to the form. Observation shows that there is one more entry in the normal application than in the \ac{SDS} application, which is quickly cycled through. But the overall notes show that they were relatively even in their cycle time. \\
Summarizing the observations when adding the data to the table, it seems that entering the data into the normal system was faster. Users quickly enter the data into the form and submit the form immediately. None of the participants ever asked for values for the status because the field was left open as a text box instead of offering a drop-down list. In comparison, two participants in the \ac{SDS} implementation application ask for preset values for status on the form. Of course, this could also be related to bad luck in the selection of participants. \\
All in all, the observations of the test runs show no decisive differences between the two systems. One interaction that stood out in both applications was the tryout pattern. Instead of reading through the text or even the navigation bar elements, users wildly click on any interaction element they find in the view. This resulted in a quick run through the test application, but shows that SaaS products should be intuitive to use.

\subsection{Measurements}
From the more subjective measurements to the objective measurements of the time required to complete the test. These measurements can confirm the subjective observations made earlier. Using the measurement points presented in Chapter \ref{text_case}, it is possible to analyze the data collected. If we take the duration between two measurement points, we obtain three time slots. 
\begin{enumerate}
    \item \textbf{Find data table}: Starting the test - see data table
    \item \textbf{Find add button}: See data table - see data add form
    \item \textbf{Add data to table}: See data add forme - submit data add form
\end{enumerate}
With the help of the measurements, it is possible to draw diagrams to understand the collected data better. One problem encountered when looking at the data is that the total duration of each run varies from run to run. A transformation with the collected data ensures that all results are comparable. In detail, the transformation calculates each duration slot relative to the total duration. All sections together result in a total percentage of 100\%. \\

With the prerequisites explained, it's time to look at the data. Starting with the measurements of the test runs on the application without \ac{SDS}.\\

\begin{figure}[htbp]
    \centerline{
    \includegraphics[width=\linewidth]{images/normal_test_data_chart.png}}
\caption{Chart of normal test data}
\label{test_data_normal}
\end{figure}
Looking at the chart in Figure \ref{test_data_normal}, the data confirms the point made in the observations. The chart is sorted by the time it took to find the data table. The first five columns show runs that were completed very quickly. As a result, data entry takes a relatively long time since typing slows down the user. A look at the end of the chart shows the long runs with a longer time on the home page. \\
The last column of the chart shows the average of all datasets combined. Once again, it's important to note that the average time spent on the distribution of finding the data table and adding data through the form is about the same. \\
The only breakout from the schema is test run 11. In this run, the relative number shows that finding the "Add +" button, i.e. showing the data table, took longer than adding the data. A look at the data shows that this was one of the longer test runs of the data set, lasting almost two minutes. Which is probably an indicator that the test user either interacted cautiously with the test application or had trouble navigating.\\
The conclusion of the data chart is that there are no surprises in the test runs for the test application without SDS. It generally confirms the previously reported observation.\\

As a next step, the elaboration deals with the results of the test run of the application with implemented SDS. Expecting to see roughly the same distribution pattern of test data.  As in Figure \ref{test_data_normal}, this chart shows the relative time windows calculated using the total time required for each run. \\

\begin{figure}[htbp]
    \centerline{
    \includegraphics[width=\linewidth]{images/sds_test_data_chart.png}}
\caption{Chart of \ac{SDS} test data}
\label{test_data_sds}
\end{figure}
At first glance, the chart in Figure \ref{test_data_sds} shows the same pattern as the normal test data chart. But in closer comparison to the normal data chart (Figure \ref{test_data_normal}), the distribution of the datasets is smoother. This indicates a more even distribution of the time spent on the different time windows. \\
A look at the average column at the bottom of the chart shows roughly the same distribution of values as the normal data. This means that finding the table of data takes, on average, the same amount of time as entering data into the form. So on average there is no difference to the test application without \ac{SDS}. \\
Looking at breakouts as previously seen on the test device, no test point can be found. The only interesting thing is that in three test runs, the search time for the "Add +" button accounted for over 15\% of the total time. A look at test runs 8, 3, and 2 in Figure \ref{test_data_sds} shows that these test runs are rather short overall. Therefore, a 4-second mouse movement on the button ends in a distribution of 15\% if the total run takes only 30 seconds. However, this is not unusual user behavior in the test application.\\
Summarizing the analysis of the data chart from the test runs of SDS data, the chart shows a smoother curve which could indicate a broader group of test users. Because every user is different when using a tool. In general, the chart shows the same usage patterns as in the normal test data. \\

In order to have comparable test data within the test sets, it was necessary to reduce the data to percentage values. There is no comparison between the two test sets. To do this, the last chart takes both average data points and compares them to each other. \\

\begin{figure}[htbp]
    \centerline{
    \includegraphics[height=10cm]{images/compare_avarages_test_data_chart.png}}
\caption{Chart of the average test data in comparison}
\label{test_data_avarages_compare}
\end{figure}
In this chart in Figure \ref{test_data_avarages_compare}, the X-axis describes an absolute scale with the total duration of the test runs on average for each test application. And surprisingly, this presentation reveals an interesting insight. \\
As said before, the distribution of the shares of time needed is roughly the same, but in terms of absolute numbers, adding data and finding the table is very different. The chart shows users who tested the application with \ac{SDS} implemented use about 15 seconds less on average than test users of the other system. \\
Finding the data table in the application with the \ac{SDS} takes about 4 seconds less than in the normal test application. This time saving is roughly equivalent to the time it takes to click the "Add +" button to navigate to the third view. But the big difference in the chart shows the bars for adding data to the form. With a difference of 8 seconds less, the test application with \ac{SDS} is significantly better.\\
In contrast to the differences presented before the third category, clicking on the "Add +" does not show any significant differences between the systems. 
\subsection{Test evaluation}
The tests were designed to prove the hypothesis of whether the SDS helps users complete a basic task faster than an application without \ac{SDS} implementation. After both test results have been presented, it is now possible to interpret the results. On the one hand, the subjective observation of each test run and, on the other hand, objective data from performance time stamps during the execution of each test run. \\
From an observational perspective, it looks like the systems behave quite similarly. As the results show, there are no particular patterns in either application. The observation therefore indicates that it makes no difference with which application the user performs the task. \\
The performance data from the test runs show a relatively similar result. In the normal test runs, there seems to be a clear division of usage patterns. Either a relatively large amount of time is spent finding the data table or a relatively large amount of time is spent adding data. Compared to the \ac{SDS} test runs, the time patterns are more balanced, with a smooth transition between long time for finding the data table and long time for data entry. This difference in patterns is indicative of more extremes in the runs for the normal version. \\
In addition to this information, the boxplot diagram provides further insights. Again, the noticeable pattern in the normal test runs is visible. The normal data has a much wider range in the boxplot than the \ac{SDS} test runs. The normal application seems to have a more random pattern of interaction than the \ac{SDS} application. But this is not all bad, the normal data run is also better in terms of minimum time. The \ac{SDS} seems to help the application create a more consistent usage pattern for the users. \\
In conclusion, the use of the \ac{SDS} helps to perform tasks more consistently and thus faster. The results of the graph show that the \ac{SDS} is generally faster. As a final step, the chi-square test is used to evaluate whether the data points just presented are statically significant.\\

To evaluate this, the recorded timestamps are subjected to a statistical hypothesis test. The chi-square test according to \citeauthor{pearson_x_1900} helps to identify whether an application is faster with SDS than without SDS. To begin the method, a null hypothesis must be defined. In this case, the null hypothesis is whether the application with SDS implemented is faster than the normal application. The result should be significant with a probability of 95\%. \cite{pearson_x_1900} \\ 
Since it is not possible to compare the data of each set of tests because they were performed by different users, a different measurement is required. Therefore, each test run is checked whether the total time required is less than 60 seconds or not. Aggregating the results down, the results table looks like this:
\begin{table}[ht]
    \centering
    \begin{tabular}{|p{0.2\linewidth} || p{0.1\linewidth}|p{0.1\linewidth}|p{0.1\linewidth}|}
        \hline
        \textbf{Total time under 60 seconds?} &\textbf{Normal}&\textbf{\ac{SDS}}&\textbf{Total} \\ \hline\hline
        \textbf{yes} & 7 & 8 & 15 \\ \hline
        \textbf{no} & 4 & 3 & 7 \\ \hline
        \textbf{Total} & 11 & 11 & 22 \\ \hline
    \end{tabular}
    \caption{\label{tab:chi-square} Aggregated test results}
\end{table}
The next step is to calculate the expected values for each variant. To do this, the sum of each test set is multiplied by the sum of the results for the met criteria of both runs. The result is then divided by the sum of all test runs. Applying this to all values in the table, the expected values look like this:
\begin{table}[ht]
    \centering
    \begin{tabular}{|p{0.2\linewidth} || p{0.1\linewidth}|p{0.1\linewidth}|p{0.1\linewidth}|}
        \hline
        \textbf{Total time under 60 seconds?} &\textbf{Normal}&\textbf{\ac{SDS}}&\textbf{Total} \\ \hline\hline
        \textbf{yes} & 7.5 & 7.5 & 15 \\ \hline
        \textbf{no} & 3.5 & 3.5 & 7 \\ \hline
        \textbf{Total} & 11 & 11 & 22 \\ \hline
    \end{tabular}
    \caption{\label{tab:chi-square-expected} Expected test results}
\end{table}
With the expected and actual values, it is now possible to calculate the chi-square value for each entry in the table. The following formula calculates the chi-square value for each entry:
\[\chi^2=\frac{(A_{jk} - E_{jk})^2}{E_{jk}}\]

Choosing $A_{jk}$ as an entry from the table of actual values and $E_{jk}$ as a value from the expected values, a new table is calculated with all $\chi^2$. The table looks like this:
\begin{table}[ht]
    \centering
    \begin{tabular}{|p{0.2\linewidth} || p{0.1\linewidth}|p{0.1\linewidth}|p{0.1\linewidth}|}
        \hline
        \textbf{Total time under 60 seconds?} &\textbf{Normal}&\textbf{\ac{SDS}}&\textbf{Total} \\ \hline\hline
        \textbf{yes} & 0.03 & 0.03 & 0.06 \\ \hline
        \textbf{no} & 0.07 & 0.07 & 0.14 \\ \hline
        \textbf{Total} & 0.1 & 0.1 & 0.2 \\ \hline
    \end{tabular}
    \caption{\label{tab:chi-square-results} Chi-squared results}
\end{table}
The total $\chi^2$ value of the test data is \texttt{0.2}. Using the chi-square distribution table, the determined probability and the degree of freedom of the table, the value for the statistical relevance is determined. For this test, the value to achieve is \texttt{3.84}. Since the total value is less than the determined value, there is no statistically significant relationship between the use of the SDS and the total time required to complete the test. \\

Thus, the test results provide a clear statement. While the test results indicate that a design system contributes to consistency in the use of an application, the evaluation of this data concludes that the results are not statistically significant and that the SDS does not affect the performance of completing tasks in an application.
% \input{chapter_5/discussion}
 
\newpage
\section{Conclusion}
This elaboration presents a possibility to perform standardization differently. A design system represents a way of creating user interfaces for ergonomically correct views. With today's technology and standardized browsers, finding a central solution that allows developers to create reusable components without the application restricted to a particular framework is no problem.  \\
Chapter \ref{saas_design_system} presents a possible concept for implementing a design system. The \acl{SDS} allows developers and designers to focus more on delivering value to their users because they do not have to worry about accessibility or best practices. In addition, the onboarding time for creators is short because a design system provides enough tools to get them started quickly. \\
With all the tools, developers can build their design system based on the \ac{SDS}. It gives them a blueprint for a design system. Not only does it extend components through the interfaces provided, but it also gives developers the ability to manipulate the existing components. The \ac{SDS} leaves enough room for creativity for their \ac{SaaS} products and brand.\\
A vital feature of the \ac{SDS} is the community aspect that surrounds it. Without it, the \ac{SDS} could be just another component library. Nevertheless, providing principles, guidelines, and a platform for designers and developers makes \ac{SDS} special. These community-built values reflect the needs of developers and designers much better than an ISO standard that is updated every two years. In addition, quick response time is essential in a fast-paced world. \\
The results of the test of the \ac{SDS} in Chapter \ref{test_results} show a specific result. Efficiency for the end users of SaaS products is crucial when developing a design system like SDS. The test result shows that a design system helps maintain an application's consistency. The visual design of both test applications is the same. Even though the observation seems to show no difference between the two applications and the data collected supports this observation. There is no statistically significant difference between the total time required for the task and the version used in the test. However, the graphs shown indicate a slight trend that the \ac{SDS} supports the consistency of the applications. Consistency is an interesting point to follow. \\
In summary, the \ac{SDS} is a valid concept for a different kind of user interface standardization for \ac{SaaS} products. Not only is it much easier to implement in existing systems, but it also provides added value for end users. Of course, this elaboration is just a concept of what such a system could look like, and much more work is needed in the real world to create a fully functional system. Nevertheless, it might be interesting to investigate these results as a first proposal further. 
\subsection{Outlook}
A possible extension of this elaboration is to adopt the idea of \ac{SDS}. Introduce new components and explore patterns. Bring this design system to life on a platform and provide a social platform for sharing current design system ideas. It is important to bring developers and designers together to work on this idea. The only driving force for this design system is community. It is the chance to simplify the standardization. \\
Of course, not all topics of such a design system can be covered in one elaboration. There are still many open questions. For example, what would a development process in such an open source project look like? There are many examples of this in open source projects, but mostly for code projects. But what about the non-code content like guidelines and design principles. Who will make those decisions? Because one person should not be the only one to determine such decisions. \\
In addition, the \ac{SDS} is not bulletproof in terms of its data. The testing was done with a relatively small group to get qualitative feedback as well. A next study would expand this to a field test with nearly hundreds of participants. This would provide better insight into the actual performance of the \ac{SDS}. In addition, with a more advanced component, it would also be possible to test different test cases. Examples would include modal interaction, user profile editing, and data deletion. \\
And just as big papers are not the Holy Grail for standardization, the same could be true for \ac{SDS}. So the question remains, how can standardization work better in the age of \ac{SaaS} products for designers, developers and their end users? And yet another solution might seem better. Right now, design systems are on the rise to standardize user interfaces in applications within organizations, and this elaboration shows that it is also possible to standardize products across an entire domain of products.  

% choose your type of citation here
%\bibliographystyle{plainnat} % author and year
%\bibliographystyle{amsalpha} % author and year short
%\bibliographystyle{alphaabbr} % author short and year
%\bibliographystyle{alpha} % author and year
%\bibliographystyle{plain} % only numbers
%\bibliographystyle{unibonn_ay} % a special file used for the uni bonn


\newpage
\singlespacing % bibliography with single spacing 
\printbibliography

%% Start of the appendix
\appendix 
%% The appendix is a chapter
\section*{Appendix}
\begin{sidewaysfigure}[tp]
    \centerline{
    \includegraphics[height=8cm]{images/normal_test_data_chart.png}}
\caption{Chart of normal test data}
\end{sidewaysfigure}
\begin{sidewaysfigure}[tp]
    \centerline{
    \includegraphics[height=8cm]{images/sds_test_data_chart.png}}
\caption{Chart of \ac{SDS} test data}
\end{sidewaysfigure}

\end{document}
 
