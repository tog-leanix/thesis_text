\subsection{User intefaces in SaaS products}
Users are still the main consumers of SaaS products. And that will not change in the next ten years. More and more products will emerge and one will be replaced by the other. Depending on what works best for the current use case. The products used in businesses could change from month to month. For users, this means they will have to learn new structures and the workflow of each software from scratch over and over again. Of course, there may be tutorials, training and documentation. But in real life, users learn how to use the software by using it. \\
A good example of this is a blog post by Snernoff \citeauthor{sernoff_website_2021} that shows how many different implementations of a navigation menu exist. No wonder the sheer amount of variants can confuse users. Remembering every interaction pattern for every piece of software used can take some time. And this example refers only to navigation menus. Usually, user interfaces consist of many different components, which leads to even more confusion when switching between different programs. \\
To solve this problem, the software programs must have similar user interfaces. For comparison: The arrangement of the pedals in motorized vehicles is always the same. The gas pedal, brake and clutch are always arranged in the same order. Thus, the user can drive any vehicle without thinking about how to operate it. Standardization once defined this arrangement, and every vehicle manufacturer must comply with it. There is also a standard for software user interfaces on how human interfaces should be designed. ISO/IEC 9241 establishes a set of rules for the ergonomic design of human interaction with computer systems. Unlike the automotive industry, there are no rules for the design of user interfaces for software products. Of course, there are exceptions, such as government websites, which must be accessible and meet certain requirements. But no one can force software manufacturers to follow a standard implementation. \\

So the goal is to get software vendors to implement some kind of standardization. Providing another standard that is a little better tailored to the use case of SaaS products may not be enough. Large standards like ISO/IEC 9241 are often far too complex to understand and don't provide pre-built components to start with. An optimal solution for implementing standardization should be easy to use for those doing the work, which are the developers. \\
In the world of user interface development on the web, many are familiar with frameworks, templates, or guidelines. They are simple and widely used in the developer community. But is it possible to have something that provides standardization for user interfaces? Such a assistant must be easy for the developer to use. In the best case, the developer doesn't need to know anything about patterns or rule sets when using it. \\
Another question that arises with standardization: Does it even make sense to have a standard for design and structure of user interfaces of SaaS products? Doesn't that lead to a world where every product looks the same? Products want to stand out from one another, on the one hand, through their capabilities, but also through their style and handling. Therefore, such an assistant must be customizable to the needs of the company. The freedom of design cannot be restricted by such a tool. The components must have the ability to adapt to the product design. \\
Assuming a tool meets all the requirements just described, i.e., it is easy to use for development and is customizable enough to stand out from other products. One final question remains: Is a user who works with a product that provides this support more productive? He should be, otherwise it makes no sense to use such a tool. \\

In examining how SaaS product companies are trying to align their user interfaces across their entire product portfolio, a new trend is emerging - design systems. So why not use a design system to bring not just one product line, but all SaaS products into alignment? The \citetitle{uswds_uswds_nodate}, for example, is already trying to unify all government-related software into one design system. So it should be possible to go a step further and standardize an entire type of application with one design system. \\
Many large SaaS companies have already developed their own design system. The \citetitle{limcaco_design_nodate} website summarizes them in a list. With 104 design systems from various SaaS vendors, this seems to be a hot topic with a lot of potential. Researching these design systems, gathering best practices, and building a base design system for all companies could lead to a new kind of standardization through a design system. \\

This makes it clear how this elaboration will approach the standardization of user interfaces with the help of design systems. The next chapter first introduces the general topic of design systems. This will be necessary later to understand how the designed basic design system got its architecture.