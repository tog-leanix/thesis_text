\subsection{\acl{SaaS}}

Speaking of technology, a new star has risen in recent years, namely \acl{SaaS}. \acl{SaaS} is a discipline of cloud computing. In general, \ac{SaaS} applications are hosted in the cloud, with which users can interact via the Internet. In this way, it is much easier to provide users with services that previously required them to install software on their desktops. This type of software often comes with a subscription model that allows users to switch from one software to another within a month. \\
\ac{SaaS} providers take care of all kinds of infrastructure issues, so the end user doesn't have to worry about those issues. In return, the end user pays for the entire service and not just for the actual application. \\
Since this type of software is available over the Internet, any connected device can access the software through a web browser. The buzzword Web 2.0 describes the evolution away from on-premise software to services delivered online. As the web browser is the main interaction tool when working with \ac{SaaS} products, the user interfaces are created web-based. \cite{hill_guide_2013}\\
\citeauthor{hill_guide_2013} summarizes the characteristics of \ac{SaaS} software as follows:
\begin{itemize}
    \item Software that is available worldwide via the Internet free of charge or for a fee
    \item Collaborative
    \item Automatic update of the software product by the manufacturer
    \item All users use the same version of the application
    \item Software product automatically scales as needed
    \item Distribution and maintenance costs are reduced due to central management in the cloud
\end{itemize}
The popularity of \ac{SaaS} products is steadily increasing. Not only because it is easy for the end user to use such software without having to install anything on his computer. But \ac{SaaS} products are also becoming more and more interesting for smaller companies. Often, payment plans are based on the number of users of the products. Therefore, even small businesses can afford software, which was not possible before. \cite{sury_software-as--service-modell_2020}\\
In the last five years alone, the number of \ac{SaaS} products used in companies has increased tenfold. \cite{stastista_saas_2021} At this high rate, the need for standardization to align software is pressing. Naturally, when thinking about alignment, a software architect thinks about connecting systems via application interfaces. These interfaces are often well documented and offer a variety of options. But what about maybe the most important interface of a software, the user interface?