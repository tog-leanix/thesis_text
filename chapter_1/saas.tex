\subsection{\acl{SaaS}}

In a world of technology, a new star has risen in recent years, namely \acl{SaaS}. \acl{SaaS} is a discipline of cloud computing. The cloud generally hosts \ac{SaaS} applications with which users can interact via the Internet. In this way, providing users with services that previously required them to install software on their desktops is much easier. In addition, this type of software often comes with a subscription model that allows users to switch from one software to another within a month. \\
\ac{SaaS} providers take care of all infrastructure issues, so the end user doesn’t have to worry about those issues. In return, the end user pays for the full service, not just the actual application. \\
Since this type of software is available over the Internet, any connected device can access the software through a web browser. The buzzword Web 2.0 describes the evolution away from on-premise software to services delivered online. The web browser is the primary interaction tool when working with \ac{SaaS} products. Most of the products build their user interfaces web-based. \cite{hill_guide_2013}\\
\citeauthor{hill_guide_2013} summarizes the characteristics of \ac{SaaS} software as follows:
\begin{itemize}
    \item Software that is available worldwide via the Internet free of charge or for a fee
    \item Collaborative
    \item Automatic update of the software products by the manufacturer
    \item All users use the same version of the application
    \item Software product automatically scales as needed
    \item Central management reduces the cost of distribution and maintenance in the 
    cloud
\end{itemize}
The popularity of \ac{SaaS} products is steadily increasing. Not only is it easy for the end user to use such software without having to install anything on his computer, but \ac{SaaS} products are also becoming more and more attractive for smaller companies. Often, the number of product users is the base for the payment plans. Therefore, even small businesses can afford software, which was not possible before. \cite{sury_software-as--service-modell_2020}\\
The number of \ac{SaaS} products used in companies has increased tenfold in the last five years. \cite{stastista_saas_2021} The need for standardization to align software is pressing at this high rate. Naturally, when thinking about alignment, a software architect thinks about connecting systems via application interfaces. These interfaces are often well documented and offer a variety of options. But what about maybe the most critical interface of the software, the user interface?