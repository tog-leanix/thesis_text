\subsection{Design System Components}
An important part of the implementation of a design system are the components. As described in chapter \ref{sds-component}, the components are created using the Lit framework. To see a complete example of a component constructed with design tokens and documented with the storybook, the example of the button component of the SDS is presented in this chapter. \\
The components use TypeScript to take advantage of custom decorators. The decorators provided by Lit further simplify the boilerplate code for creating a web component. \\
\lstinputlisting[linerange={1-4},firstnumber=1,caption={Initialization of SDS button component},label=ButtonInit]{../Code/src/components/button.component.ts}
In listing \ref{ButtonInit}, the component is initialised with the \texttt{@customElement} decorator by passing the tag name as a string and appending it to an ES6 class. For it to work properly, the class must extend the LitElement class. When everything has been implemented as described, the web component is registered and can be used with the defined tag, in this case \texttt{<saas-button>}. \\
In order to see something when the web component just created is used, it must implement the render method. This method expects a \texttt{TemplateResult}. 
\lstinputlisting[linerange={19-26},firstnumber=19,caption={Rendering of SDS button component},label=ButtonRender]{../Code/src/components/button.component.ts}
Lit element provides the import of a \texttt{html} string literal that can be used to construct the \texttt{TemplateResult} expected by the render function. With this functionality, Lit provides an efficient way to respond to variable changes and intelligently update the DOM. \\
Properties also make use of custom decorators and declare properties on web components by writing \texttt{@property} in front of a class attribute (line 19-20, Listing \ref{ButtonRender}). Thus, Lit Element implements change detection and adds automatic type conversion. For a detailed description of the capabilities of this decorator, see the documentation. Whenever the property value of the created button web component changes, the constructed template string reacts to these changes and updates the displayed template accordingly. \\
Last but not least, the web component must use the defined design tokens. Since these tokens are defined by CSS variables, they can simply be consumed as follows: 
\lstinputlisting[linerange={5-18},firstnumber=5,caption={Styles of SDS button component},label=ButtonStyles]{../Code/src/components/button.component.ts}
The Lit Element framework implements a static property of the \texttt{styles} to its classes. To make it work properly, a string literal \texttt{css} is used to convert a string into the required \texttt{CSSResult} type. In this way it would be possible to consume design tokens via inout properties, but to simply manipulate design tokens in one place, the components of SDS will use CSS variables. In this way, it is possible to customise certain tokens throughout the design system by adapting these tokens. Each component in the SDS will use the defined tokens. Line 7-9 in listing \ref{ButtonStyles} shows an example of the use of design tokens for the button component in the SDS. \\
