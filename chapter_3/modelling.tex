\subsection{Architecture}
Modeling the architecture of a system that supports customization of SaaS product user interfaces is not a trivial task. Often, user interfaces for products are developed only to support their purpose. Finding common ground for multiple products can be difficult.  \\
Design systems define a common place where the company's products can align. Why is it not possible to develop a central system to create a common idea for user-friendly SaaS products? Not only do well-designed components help align, but a design system foundation with guidelines and principles helps developers and designers create a good product. \\
To this end, 10 different design systems of well-known SaaS products, as well as design systems with a shared purpose, were considered. See table \ref{tab:design_systems_in_the_wild} for more information.
\newpage
\begin{table}[h!]
\begin{tabular}{|p{0.2\linewidth} | p{0.7\linewidth}|}
\hline
 \textbf{Name} & \textbf{Description} \\ \hline
Service Now  & Platform design system.  Guidance to create components and upload them to the platform. \\ \hline
Adobe Spectrum  & User centralised Design System. Many well designed components with matching guide to deliver a great experience. Built in web components and react components. \\ \hline
Zendesk Garden & Basic Design System with guidelines, components and patterns. Tailwindcss integration. Built in react components. \\ \hline
Atlassian Design System & Design System connected with company values. A lot of guides on how to use designs, components and to write content. Includes also employee motivation. Built in react components. \\ \hline
Base Web  & Open source Design System. Used by Uber. Providing a blog and guides on how to use the base design system. Design System intended to be used as baseline and should be overwritten when used. No principles or values included. \\ \hline
SAP Fiori  & Standard Design System. Focus on accessibility and multiple device support. Including many themes for different applications. Delivers a toolkit to better use the design system as a designer.  \\ \hline
GOV UK Design System  & Not really a Design System. Missing guidelines and principles. Externals can propose changes. Providing CSS classes for HTML elements.  \\ \hline
Lightning Design System & Design System to support developers and designer at their work. 4 principles with a clear message to align every user. Guidelines and best practices on many topics.  Components are built with pure CSS classes. \\ \hline
Google Matrial Design & Open source Design System. Providing the user with design principles which helps to understand the usage of the Design System. Material Design provides only components and no patterns. Blogs and further resources are helping additionally to the guidelines. Components are built with pure CSS classes. \\ \hline
Pluralslight Design System & Design System without principles and guidelines. For the moment only components are present. Providing a workflow for developers and designers to contribute to the Design System.  Only few patterns. Built with react components.  \\ \hline
\end{tabular}
\caption{\label{tab:design_systems_in_the_wild} Overview of 10 existing Design Systems}
\end{table}