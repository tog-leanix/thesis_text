\subsection{Architecture}
Modelling the architecture of a system that supports customization of SaaS product user interfaces is not a trivial task. Often, user interfaces for products are developed only to support their purpose. Finding common ground for multiple products can be difficult.  \\
Design systems define a common place where the company's products can align. Why is it not possible to develop a central system to create a common idea for user-friendly SaaS products? Not only do well-designed components help align, but a design system foundation with guidelines and principles helps developers and designers create a good product. \\
To this end, 10 different design systems of well-known SaaS products, as well as design systems with a shared purpose, were considered. 
\begin{table}[!ht]
\begin{tabular}{|p{0.2\linewidth} | p{0.7\linewidth}|}
\hline
 \textbf{Name} & \textbf{Description} \\ \hline
Service Now  & Platform design system.  Guidance to create components and upload them to the platform. \\ \hline
Adobe Spectrum  & User centralised design system. Many well designed components with matching guide to deliver a great experience. Built in web components and react components. \\ \hline
Zendesk Garden & Basic design system with guidelines, components and patterns. Tailwindcss integration. Built in react components. \\ \hline
Atlassian Design System & Design System connected with company values. A lot of guides on how to use designs, components and to write content. Includes also employee motivation. Built in react components. \\ \hline
Base Web  & Open source design system. Used by Uber. Providing a blog and guides on how to use the base design system. Design System intended to be used as baseline and should be overwritten when used. No principles or values included. \\ \hline
SAP Fiori  & Standard design system. Focus on accessibility and multiple device support. Including many themes for different applications. Delivers a toolkit to better use the design system as a designer.  \\ \hline
GOV UK Design System  & Not really a design system. Missing guidelines and principles. Externals can propose changes. Providing CSS classes for HTML elements.  \\ \hline
Lightning Design System & Design System to support developers and designer at their work. 4 principles with a clear message to align every user. Guidelines and best practices on many topics.  Components are built with pure CSS classes. \\ \hline
Google Matrial Design & Open source design system. Providing the user with design principles which helps to understand the usage of the design system. Material Design provides only components and no patterns. Blogs and further resources are helping additionally to the guidelines. Components are built with pure CSS classes. \\ \hline
Pluralslight Design System & Design System without principles and guidelines. For the moment only components are present. Providing a workflow for developers and designers to contribute to the design system.  Only few patterns. Built with react components.  \\ \hline
\end{tabular}
\caption{\label{tab:design_systems_in_the_wild} Overview of 10 existing design systems}
\end{table}
As can be seen in table \ref{tab:design_systems_in_the_wild} fot these 10 examples of design system, the interpretation of one can vary. \\
A good reference for a design system with a suitable use case is the Base Web Design System. Its purpose of providing a base of styles and functional components helps developers customise for their products. Also, the fact that this design system is open source underlines that this design system has been developed by the community and is not controlled by a corporate design team. Built-in accessibility is also a requirement that must be present in a common design system. The instructions on how to extend and use the design system are also a perfect reference. \\
But there are also disadvantages. The Base Web Design System lacks guidelines and principles that are crucial for a design system. A look at Google's Material Design shows that even an open-source and versatile design system can have design principles. Design principles help developers get an idea of how to develop and design with the system. Therefore, design principles are indispensable in a design system and should not be missing. \\ \newpage
Another reason, is the use of React for creating the components in the Base Web Design System. As a design system that should be used by everyone as a standard for the implementation of SaaS products, it is therefore unsuitable. As developers are expected to use React as a frontend library. Looking at other examples such as the GOV UK Design System or Salesforce's Lightning Design System shows that it is possible to create components using web standards that can be used by anyone without having to use a library. \\
With these requirements, an architecture can be drawn as can be seen in figure \ref{architecture_sds}. \\
\begin{figure}[htbp]
\centerline{\includegraphics[width=\linewidth]{images/architecture_sds.png}}
\caption{Architecture of SaaS Design System}
\label{architecture_sds}
\end{figure}
According to the description of a design system presented in chapter 2, the SaaS design system can be roughly divided into four parts. \\
\subsubsection*{Design Principles}
In terms of design principles, the SDS strives to keep them lean and easy to understand, as the design system is intended to serve as a basis for other design systems. The \textbf{Simple} principle states that component design should not have unnecessary styles or features that make it difficult to extend. \\
Many products strive to implement accessibility in their products. With the \textbf{Accessible} principle, SDS emphasises the importance of accessible user interfaces. This is not only important in terms of inclusion, but also helps accessibility in the overall user experience for all users. \\
The third and final principle is \textbf{Solid}. As stated earlier, the SDS should be a foundation for other design systems to build upon. Therefore, the importance of a stable and consistent API is very important. For this reason, the design system has deliberately chosen \textbf{Solid} as the third and final principle. \\
\subsubsection*{Guidelines}
The SDS guidelines are based on the design principles just presented. In addition to the core guidelines on extension, accessibility and basic use, there are also guidelines that focus on contributions and collaboration. As this is an open source system, as many people as possible should be able to work on it. \\
The extension guidelines address how to integrate SDS as the basis for a company's own design system. It shows developers and designers how to create their own from the components provided. \\
Since accessibility is also a design principle, there must be a guideline that defines what the SDS means by accessibility. It should give the user a definition and sources for accessibility. But also self-designed components should have a guideline that helps users to implement accessibility. After this guideline, the user should have no more questions about accessibility. \\
As a third guideline, the SDS will support the user in using the system. This guide could also be seen as an entry guide and will cover the basics. Importing the design system, proper bootstrapping and guidance on configuring the system. This may seem self-explanatory, but lacking these often prevents users from using the system effectively. The user guide should be as simple as possible and cover every small step needed to get started with the SDS. \\
One goal of this design system is to be developed by the community for the community. However, this can quickly get out of hand if everyone contributes without any guidance. Therefore, it is important to introduce some from the beginning. This guide provides guidance on how to contribute to the component library, but also on how to enforce changes to the guidelines and principles. As this design system is not set in stone, there should be opportunities to change and adapt everything. What this will look like in the end will evolve over time. Some ideas could be a voting process or an RFC (source) process, as is common in the software industry. \\
To achieve high interactivity in such a design system, some design systems introduce blogs and forums for knowledge exchange. In this way, users can connect, discuss and contribute to ideas to further improve the design system. The most important point is the moderation of such an interaction platform. A well-moderated blog and forum will further enhance the community around the SDS. \\
Overall, it can be said that the guidelines for the SDS are aimed at building a community around the design system to contribute and enable users to create a community design system. \\
\subsubsection*{Components}
% The SDS guidelines are based on the design principles just presented. In addition to the core guidelines on expansion, accessibility and basic use, there are also guidelines that focus on contributions and collaboration. Since it is an open source system, as many people as possible should be able to work on it. 
% The extension guideline is tackling the need on how to integrate the SDS as baseline for companies own design system. Guiding developers and designers how to create their own components out of  the provided ones. 
% Since accessibility is also a design principle there has to be a guideline which determines what the SDS understands under accessibility. Giving the user a definition and sources for accessibility. But also a guideline to provide the user with a definition of done for self-constructed components to align with SDS components. Following this guideline the user should not have any questions left regarding accessibility. 
% As third guideline the SDS will support user on the usage of the system. This guideline could be also seen as getting started guide and will cover basics. Importing the design system, bootstrapping it correctly and guide the user on how to configure it. This might seem self-explanatory but often prevent users to use the system. The usage guideline should be as easy as possible and should cover every little step needed to get started with the SDS.
% A goal of this design system is to be build by the community, for the community. But this can get quickly out of hand if everyone is contributing without any guideline. Therefore it is important to introduce some from the start. This guideline includes guidance on contribution to the component library, but also how to push for changes for guideline and principles. Since this design system is not set in stone there should be ways to change and adapt everything. How this will look like in the end will develop over time. Some ideas could be a voting process or a RFC (source) process which is commonly used in software industry. 
% To achieve a high interactivity on such a design system some design system introduce blogs and forum to share knowledge. In this ways users can connect, discuss and contribute to ideas and improve the design system even better. Key point is the moderation of such kind of interaction platform. A good moderated blog and forum will improve the community around the SDS even more.
% Overall it can be seen that the guideline for the SDS are oriented to build a community around the design system to contribute and give the users the opportunity to shape a communities design system.
