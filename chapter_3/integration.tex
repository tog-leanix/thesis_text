\subsection{Design System Integration}
The last step to complete the user story of the SDS is to integrate the system into an application. This chapter explains how to package the SDS and load it into a desired application. It is important that the integration is simple so as not to discourage developers from using the design system. \\

The build process is the first part that is important to understand the integration workflow. Webpack is the build tool used by SDS. As shown in Figure 3 in Chapter 1, the components of the design system are built using Typescript, including the Lit framework and using SCSS for styling. Since Typescript and SCSS are not supported by the browser, they must be processed beforehand. Therefore, Webpack comes into play to compile the code. \\
Webpack uses a configuration file, often called webpack.config.js, to describe the steps by which the code is compiled. This file describes rules that tell Webpack what to do with which files that come into the build pipeline. These rules for SDS look like this:\\
\lstinputlisting[linerange={23-41},firstnumber=23,caption={SDS Webpack rules},label=WebpackRules]{../Code/webpack.config.js}
The rules define regular expressions that are used to assign files with their extensions to the corresponding compilers, which are called loaders in the Webpack world. The loaders used here are build-in. But there are also custom ones that can be integrated into a build process. When using loaders, it is sufficient to include the loader string in the use property of a rule object. For example, in line 25-26, Listing \ref{WebpackRules}, the Typescript loader is matched with the regular expression Typescript to process Typescript files. The same pattern is used in line 34-35, Listing \ref{WebpackRules} to match SCSS files with the default style loader that comes with Webpack.