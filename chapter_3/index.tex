\newpage
\section{The \acl{SDS}}\label{saas_design_system}
This chapter presents the implementation of the newly created design system as a standard for the web user interfaces for SaaS products. \\
Modeling the architecture of a new design system is the first step. Looking at existing design systems and trying to understand how they work and extract requirements in order to develop a common system that fits most products in the SaaS world. \\
Then, this chapter presents the actual implementation of a component, including the technology stack used and the creation process. \\
The chapter concludes with an explanation of the build chain. How the build process bundles all the components and atoms to make them usable for any project. At the end, an example shows the possibilities to integrate the finished bundle into existing software.\\
Modeling the architecture of a system that supports customization of SaaS product user interfaces is not a trivial task. Most of the time, user interface design serves only one product. However, the goal of \ac{SDS} is to find common ground for multiple products in one domain. \\
Design systems define a common place where the company's products can align. Why is it impossible to develop a central system to create a common idea for user-friendly SaaS products? Not only do well-designed components help align, but a design system foundation with guidelines and principles helps developers and designers create a good product. \\

\subsection{Requirements}
Defining requirements for a design system that will serve as the foundation for an entire product area is no easy task. In the world of SaaS products, there is no central place for best practices and guidelines. For this reason, this chapter looks at both functional and technical requirements. Functional requirements take the perspective of the various stakeholders in the system and model what requirements they might have for the system. For technical requirements, on the other hand, an exploration of ten design systems helps to identify the most important technical requirements for the \ac{SDS}.
\subsubsection{Functional requirements}
% Add use cases here
Functional requirements describe the necessary functionality of a software system. Hull et al. suggest modeling functional requirements using use case diagrams. These diagrams show the relationship between the acting roles and the presented system.\cite{hull_requirements_2011} \\
In the case of \ac{SDS}, there are two different types of interactions. There is an active interaction and a passive interaction. \\
Passive interaction means that the design system is implemented by any web application and the actor uses the design system passively by interacting with the web application. Since the web application uses components and policies of \ac{SDS}, the actor interacts with the design system under the hood. \\
Active interaction, on the other hand, is when an actor interacts directly with the design system. Directly means that he accesses the design system documentation and/or its components and uses them to implement or design new applications. \\
Table \ref{tab:stakeholders_sds} provides a brief overview of the actors and their type of interaction with the \ac{SDS}.
\begin{table}[!ht]
    \begin{tabular}{|p{0.2\linewidth} | p{0.15\linewidth}| p{0.5\linewidth}|}
    \hline
     \textbf{Actor} &\textbf{Interaction} & \textbf{Description} \\ \hline
     Collaborator & Active & Collaborators are designers and developers who want to contribute to the design system. Often, these types of contributors are also collaborators in other open source projects. They are not paid for their work on the design system.  \\ \hline
     Users & Active & Developers and designers who use the design system to develop new products. Basically, they are the customers of the design system. It is important for them to be supported in the creation of their products. The design system makes their work easier.  \\ \hline
     End users & Passive & End users are all people who use SaaS products in their daily work. They do not actively interact with the design system, but with a web application that implements the design system. They do not need to know whether the web application they use implements the design system. \\ \hline
     
    \end{tabular}
    \caption{\label{tab:stakeholders_sds} Stakeholder of \ac{SDS}}
\end{table}
The personas presented are an extraction of personas that are most likely to interact with the design system. The selected personas help to create use cases that describe the functional requirements for the \ac{SDS}. By looking at the system from the position of each actor, it is possible to formulate requirements for a design system that will serve as the basis for a SaaS product. The type of interaction with the design system separates the following use case diagrams, starting with the passive interaction of the personas. \\

\begin{figure}[htbp]
    \centerline{\includegraphics[width=\linewidth]{images/use_case_diagram_passive.png}}
    \caption{Use case diagram passive interaction with \ac{SDS}}
    \label{use_case_passive}
\end{figure}
Figure \ref{use_case_passive} shows the passive interaction with the SDS via a web application that implements the design system. The only actor in this case is the end users, described in Table \ref{tab:stakeholders_sds}. \\ 
Actors passively interact with the design system through the web application, which leads to five use cases. There could be more use cases for the end users, but for now these are the five most important to consider. \\
\begin{enumerate}
    \item The design system should allow the end user to navigate through the applications in the same way. This means that every application that implements the patterns and components of SDS has the same navigation structure. Therefore, end users do not have to get used to a new navigation structure when they switch to another application.
    \item Controls in web applications should have the arrangement by different applications. The end user does not have to learn how to submit a form or find the settings for their user profile.
    \item Call-to-actions should have the same look and feel from application to application. This way, buttons on pages can be found quickly. This will help the end user increase productivity.
    \item End users may be temporarily or permanently unable to use the mouse. Therefore, it is important that they can navigate through an application using the keyboard. The design system should be aware of this requirement.
    \item End users may not be able to see the website temporarily or permanently. Therefore, it is important that they can use a screen reader when interacting with the product. The design system should implement components so that they can be read by a screen reader.
\end{enumerate}
This completes the initial use cases for passive use of the SDS. \\

Compared to passive interaction with the design system, there are two types of actors in active interaction. Table \ref{tab:stakeholders_sds} shows both the collaborators and the users of the active interaction. Both actors create a variety of use cases. Some of them are used by only one actor, while others are shared. Figure \ref{use_case_active} shows a use case diagram for active interaction with the design system. \\
\begin{figure}[htbp]
    \centerline{\includegraphics[width=\linewidth]{images/use_case_diagram_active.png}}
    \caption{Use case diagram active interaction with \ac{SDS}}
    \label{use_case_active}
\end{figure}
The visual representation of the use cases allows them to be separated for further explanation. Therefore, the use cases just presented are divided into three different categories. Use cases that are assigned to collaborators, use cases that are assigned to users, and use cases that are assigned to both actors. \\

Starting with the foundation for \ac{SDS}, the collaborator use cases.
\begin{enumerate}
    \item Collaborators want to manage, contribute and program new components of the design system's component library. It should be easy to contribute to such a design system.
    \item Collaborators want to manage and contribute to the patterns of a design system. Patterns need a process to define new patterns and discuss existing ones to improve them.
    \item Collaborators have the ability to create new design tokens. Design tokens define the basis for all components and patterns in the design system. 
    \item Collaborators want to discuss and adjust the design principles for the \ac{SDS} design system. The principles should not change too much, so an appropriate process should be defined to have a reasonable management process. 
\end{enumerate}

The second category of use cases are use cases that relate only to the users of the design system. As described earlier, users only want to use the design system's provided assets and do not want to edit or create content on the design system itself.
\begin{enumerate}
    \item Users want to use the standardized components provided by the design system. The actions required to implement the component in the user's application should be as simple as possible.
    \item Users want to use the best practice patterns provided by the design system. The documentation and the guide according to the pattern should lead the users to an easy implementation.
    \item Users want to use the design tokens provided by the design system. Design tokens not only help style the components and patterns provided by the design system, but also support components and structures created by the user's application. This results in a consistent style throughout the application, whether or not the design system components are used.
    \item Users do not want to destroy their running applications that implement the design system. Therefore, it is important to provide for versioning of the design system. They also need to be able to use older versions of the design system when needed and decide when to upgrade the application to the latest version.
\end{enumerate}
Last but not least, we are left with use cases that involve both actors and users. These use cases are special because they address both parties. A detailed look at these cases shows what they are all about.
\begin{enumerate}
    \item Active stakeholders should have the opportunity to exchange ideas and thoughts on a platform such as a shared design system. New content for the design system can emerge from these interactions. With a kind of communication platform for the design system, a community can form around it. 
    \item Both stakeholder groups want to follow the design principles when creating new components, whether for their applications or for the design system. This doubles the importance of design principles within a design system. The goal of a design system is to make them simple accessible and understandable.
    \item Active actors on the SDS want to be informed about the latest trends and best practices in SaaS applications. Therefore, a design system should provide an option to publish news on these topics. This will lead to more interaction on the design system and can motivate stakeholders to discuss on the communication platform.
\end{enumerate}

This summarizes the functional requirements for the SDS. The number of use cases for such a design system is much larger, but this elaboration focuses on the most important ones. In the future, when the system is in use, there could be many more use cases. However, this should be an iterative process driven by the community that forms around the system.

\subsubsection{Technical requirements}
In addition to functional requirements, there are also technical requirements. Technical requirements describe the technical details of a design system. This elaboration defines the technical requirements looking into SaaS product design systems as well as design systems that share a common purpose. With this data, it is possible to extract technical requirements when implementing such a design system.\\
The goal of the research is to identify best practices from the SaaS world and understand the needs of companies that use design systems. In addition, when looking at broader design systems with a common purpose, there is a great opportunity of finding a reference design system. Table \ref{tab:design_systems_in_the_wild} lists the studies of ten design systems with a brief summary.
% \newpage
% \begin{table}[!ht]
\begin{longtable}{|p{0.2\linewidth} | p{0.7\linewidth}|}
\hline
 \textbf{Name} & \textbf{Description} \\ \hline
ServiceNow \cite{servicenow_servicenow_nodate}  & Platform design system.  Guidance to create components and upload them to the platform. \\ \hline
Adobe Spectrum  \cite{spectrum_adobe_spectrum_nodate} & User centralised design system. Many well designed components with matching guide to deliver a great experience. Built in web components and react components. \\ \hline
Zendesk Garden \cite{zendesk_garden_zendesk_nodate} & Basic design system with guidelines, components and patterns. Tailwindcss integration. Built in react components. \\ \hline

Atlassian Design System \cite{atlassian_design_system_atlassian_nodate} & Design System connected with company values. A lot of guides on how to use designs, components and to write content. Includes also employee motivation. Built in react components. \\ \hline
Base Web  \cite{base_base_nodate} & Open source design system. Used by Uber. Providing a blog and guides on how to use the base design system. Design System intended to be used as baseline and should be overwritten when used. No principles or values included. \\ \hline
SAP Fiori  \cite{sap_fiori_nodate} & Standard design system. Focus on accessibility and multiple device support. Including many themes for different applications. Delivers a toolkit to better use the design system as a designer.  \\ \hline
GOV UK Design System  \cite{govuk_govuk_nodate} & Not really a design system. Missing guidelines and principles. Externals can propose changes. Providing CSS classes for HTML elements.  \\ \hline
Lightning Design System \cite{lightning_design_system_lightning_nodate} & Design System to support developers and designer at their work. 4 principles with a clear message to align every user. Guidelines and best practices on many topics.  Components are built with pure CSS classes. \\ \hline
Google Matrial Design \cite{google_material_2022} & Open source design system. Providing the user with design principles which helps to understand the usage of the design system. Material Design provides only components and no patterns. Blogs and further resources are helping additionally to the guidelines. Components are built with pure CSS classes. \\ \hline
Pluralslight Design System \cite{pluralsight_ds_nodate} & Design System without principles and guidelines. For the moment only components are present. Providing a workflow for developers and designers to contribute to the design system.  Only few patterns. Built with react components.  \\ \hline
\caption{\label{tab:design_systems_in_the_wild} Overview of 10 existing design systems}
\end{longtable}
% \end{table}
The analysis of the design systems listed above derives three central technical requirements. These requirements serve as a guide for the development of the design system.
\begin{enumerate}
    \item A good reference for a design system with a suitable use case is the Base Web Design System. Its purpose of providing a base of styles and functional components helps developers to extend the design system and create their own for their products. Also, the fact that this design system is open source underlines that the community is developing this design system, not a corporate design team. As a shared design system, it focuses particularly on accessibility and internationalization. The tutorial explains how to extend and use the design system, which is a perfect reference model for \ac{SDS}. 
    \item A look at Google's Material Design shows that even an open source and versatile design system can have design principles. Design principles help developers get an idea of how to develop and design with the system. Therefore, design principles are essential for a common design system like SDS.
    \item Another good assumption for \ac{SDS} is adaptability. Examination of some design systems shows that some of them use a front-end framework to develop their components. This ties the users of the design system to a framework. However, design systems such as GOV UK and Salesforce's Lightning Design System do not use frameworks. Their components consist solely of web standards such as HTML and CSS. Therefore, the exclusive use of web standards is a requirement for the \ac{SDS}.
\end{enumerate}

In all the design systems listed in Table \ref{tab:design_systems_in_the_wild}, there are many more requirements that could be useful for a design system that serves as a basis for others. Due to the limitation of this elaboration, only three main aspects have been presented above.\\





\subsection{Architecture}
Modeling the architecture of a system that supports customization of SaaS product user interfaces is not a trivial task. Most of the time, user interface design serves only one product. However, the goal of \ac{SDS} is to find common ground for multiple products in one domain. \\
Design systems define a common place where the company's products can align. Why is it impossible to develop a central system to create a common idea for user-friendly SaaS products? Not only do well-designed components help align, but a design system foundation with guidelines and principles helps developers and designers create a good product. \\

Discovering ten different design systems from popular SaaS products, as well as design systems with a common purpose, help model a new design system. The goal is to identify best practices from the SaaS world and understand the needs of companies using design systems. Table \ref{tab:design_systems_in_the_wild} reflects the idea of a design system for any business or purpose. The idea varies from design system to design system. \\
\begin{table}[!ht]
\begin{tabular}{|p{0.2\linewidth} | p{0.7\linewidth}|}
\hline
 \textbf{Name} & \textbf{Description} \\ \hline
ServiceNow \cite{servicenow_servicenow_nodate}  & Platform design system.  Guidance to create components and upload them to the platform. \\ \hline
Adobe Spectrum  \cite{spectrum_adobe_spectrum_nodate} & User centralised design system. Many well designed components with matching guide to deliver a great experience. Built in web components and react components. \\ \hline
Zendesk Garden \cite{zendesk_garden_zendesk_nodate} & Basic design system with guidelines, components and patterns. Tailwindcss integration. Built in react components. \\ \hline
Atlassian Design System \cite{atlassian_design_system_atlassian_nodate} & Design System connected with company values. A lot of guides on how to use designs, components and to write content. Includes also employee motivation. Built in react components. \\ \hline
Base Web  \cite{base_base_nodate} & Open source design system. Used by Uber. Providing a blog and guides on how to use the base design system. Design System intended to be used as baseline and should be overwritten when used. No principles or values included. \\ \hline
SAP Fiori  \cite{sap_fiori_nodate} & Standard design system. Focus on accessibility and multiple device support. Including many themes for different applications. Delivers a toolkit to better use the design system as a designer.  \\ \hline
GOV UK Design System  \cite{govuk_govuk_nodate} & Not really a design system. Missing guidelines and principles. Externals can propose changes. Providing CSS classes for HTML elements.  \\ \hline
Lightning Design System \cite{lightning_design_system_lightning_nodate} & Design System to support developers and designer at their work. 4 principles with a clear message to align every user. Guidelines and best practices on many topics.  Components are built with pure CSS classes. \\ \hline
Google Matrial Design \cite{google_material_2022} & Open source design system. Providing the user with design principles which helps to understand the usage of the design system. Material Design provides only components and no patterns. Blogs and further resources are helping additionally to the guidelines. Components are built with pure CSS classes. \\ \hline
Pluralslight Design System \cite{pluralsight_ds_nodate} & Design System without principles and guidelines. For the moment only components are present. Providing a workflow for developers and designers to contribute to the design system.  Only few patterns. Built with react components.  \\ \hline
\end{tabular}
\caption{\label{tab:design_systems_in_the_wild} Overview of 10 existing design systems}
\end{table}

A good reference for a design system with a suitable use case is the Base Web Design System. Its purpose of providing a base of styles and functional components helps developers to extend the design system and create their own for their products. Also, the fact that this design system is open source underlines that the community is developing this design system, not a corporate design team. As a shared design system, it focuses particularly on accessibility and internationalization. The tutorial explains how to extend and use the design system, which is a perfect reference model for \ac{SDS}. \\

But there are also disadvantages. The Base Web Design System lacks guidelines and principles that are crucial for a design system.
A look at Google's Material Design shows that even an open-source and versatile design system can have design principles. Design principles help developers get an idea of how to develop and design with the system. Therefore, design principles are essential for design systems. \\

Another reason why the Base Web Design System is not a perfect model is the use of React to create the components. The design system forces developers to use React as a front-end library in their products. This is not an option for SDS to bind developers to a single front-end framework. Therefore, this characteristic of the Base Web Design System is unsuitable. Other examples such as the GOV UK Design System or Salesforce's Lightning Design System show that it is possible to create components using only web standards. \\

With these requirements, Figure \ref{architecture_sds} draws an example architecture of what the SDS may look like.\\
\begin{figure}[htbp]
\centerline{\includegraphics[width=\linewidth]{images/architecture_sds.png}}
\caption{Architecture of SaaS Design System}
\label{architecture_sds}
\end{figure}

The architecture model shown above divides the \acl{SDS} into four parts. The details of these parts are explained in the next sections.
\subsubsection{Design Principles}
In terms of design principles, the \ac{SDS} strives to keep them lean and easy to understand. This helps the design system to serve as a basis for other design systems. \\

The \textbf{Simple} principle states that component design should not have unnecessary styles or features that make it difficult to extend. This principle underlines the idea of keeping the entire design system lean. Developers understand lean descriptions and clear documentation much better than reading a wall of text. \\

Many products strive to implement accessibility in their products. With the \textbf{Accessible} principle, \ac{SDS} emphasizes the importance of accessible user interfaces. This emphasizes not only significant in terms of inclusion, but also helps accessibility in the overall user experience for all users. Because \textbf{Accessibility} doesn't just come into play when it relates to disabilities, it helps the application with the overall user experience. \\

The third and final principle is \textbf{Solid}. As stated earlier, the \ac{SDS} is a foundation for other design systems to build upon. Therefore, the importance of a stable and consistent API is paramount. Components and patterns should not change regularly. Versioning allows developers to choose their desired version of the \ac{SDS} without having to adapt. For this reason, the design system has deliberately chosen \textbf{Solid} as the third and final principle. \\

The first iteration of the \acl{SDS} principles provides a good foundation on which to build. As described in Chapter \ref{design_principles}, finding the right design principles will take a few iterations to get right. But by starting with \textbf{Simple}, \textbf{Accessible}, and \textbf{Solid}, designers and developers will find the right way to use this design system.
\subsubsection{Guidelines}
The design principles just presented form the basis for the \ac{SDS} guidelines. In addition to the core guidelines on extension, accessibility, and essential use, some guidelines focus on contributions and collaboration. As this is an open-source system, as many people as possible should be able to work on it. \\

The extension guidelines address how to integrate \ac{SDS} as the basis for a company's own design system. It shows developers and designers how to create their own from the components provided. The goal of such a guide is to help system developers use the SDS as a foundation. The guide helps developers and designers to build their system on the SDS instead of introducing complex integrations. \\

Since accessibility is also a design principle, a guideline must define what the \ac{SDS} means by accessibility. The guideline should give the user a definition and sources for accessibility. But there should also be a manual for self-designed components to help users implement accessibility. Understanding this guide, users should no longer have accessibility questions when designing new interfaces. \\

As a third guideline, the \ac{SDS} will support the user in using the system. This guide could also be seen as an entry guide and will cover the basics. Importing the design system, proper bootstrapping, and guidance on configuring the system. Such a guideline may seem self-explanatory, but the lack of it often prevents users from using the system. The usage guide should be as simple as possible and cover every small step needed to get started with the \ac{SDS}. \\

One goal of this design system is to be developed by the community for the community. However, this can quickly get out of hand if everyone contributes without guidance. Therefore, it is crucial to introduce some from the beginning. This guide guides how to contribute to the component library and enforce changes to the guidelines and principles. \\
As this design system evolves over time, there should be opportunities to change and adapt. What this will look like in the end will evolve over time. What this will look like in the end will evolve. Some ideas could be a voting process or an RFC process, as is standard in the software industry. \\


Some design systems introduce blogs and forums for knowledge exchange to achieve high interactivity in a design system. In this way, users can connect, discuss and contribute to ideas to further improve the design system. A well-moderated blog and forum will further enhance the community around the \ac{SDS}. \\
Since the \ac{SDS} won't be a product but an open source project, forums and blogs will be the marketing platform to spread the idea of the design system. The goal is to build a strong community around the \ac{SDS}. So that some of the advocates end up contributing to the actual design system. \\

In summary, the guidelines of the \acl{SDS} are about creating a community around the design system. Developer and designers should not only using it for their desired goal, but see the potential to collaborate for the bigger picture behind \ac{SDS}.


\subsubsection{Components} \label{sds-component}
Without well-assembled components, design systems cannot exist. Therefore, choosing the right technology package for building components is very important. In the case of \ac{SDS}, one of the most essential requirements is that the system is usable independently of the frontend framework used. To achieve this, \ac{SDS} uses the web components supported in almost all modern browsers. As it is possible to create web components without importing libraries or frameworks, it is for \ac{SDS}. \citep{mdn_web_component_nodate} \\

Creating web components natively can be complicated. The Lit Element framework helps developers build web components by eliminating some of the pitfalls of implementing web components from scratch. With a focus on ease of understanding, intelligent DOM updates, and a small package size (5 KB), Lit is a perfect addition for creating components for design systems based on standard web components. \citep{lit_nodate} \\

To further assist developers, \ac{SDS} uses Typescript, a superset of Javascript. It extends Javascript with types and interfaces. Typescript must be compiled into Javascript for the browser to understand, but this allows the developer to find errors much faster because it fails at compile time rather than at runtime. \citep{microsoft_typescript_nodate} \\

The \ac{SDS} uses custom CSS variables to use and provide design tokens for colors or spacing. The design system imports the design tokens into the root element during bootstrapping. In the documentation of the design system exists a page with an overview of all design tokens. \citep{mdn_css_vars_nodate} \\

Last but not least, users must have access to the documentation of the components and tokens. Storybook is the documentation tool for the \ac{SDS}. It has many built-in functions that are useful for documenting components. With MDX, the combination of Markdown templating (MD) and code injection via JSX, it is possible to write fluid documentation without having to jump back and forth between files while documenting components. \citep{otander_markdown_2017} \\

With this technology stack, \ac{SDS} provides users with highly reusable web components that are easy for users to access but also easy for contributors to develop.


\subsubsection{Patterns}
An essential part of a design systems are patterns. Patterns try to combine the capabilities of a design system with the design of user interfaces. In the case of \ac{SDS}, patterns help developers automatically apply best practices and web standards without having to read an entire text. \\

The combination of standard HTML elements and components from SDS creates patterns in the SDS. These are easily accessible to the developer by copying and pasting code into the desired application that has implemented the SDS. Additional documentation helps the developer to avoid using patterns where they do not belong. \\

SDS patterns come with special documentation that allows for customization and redesign. The documentation enables developers to fulfill their requirements for their own design system.\\

As with the component documentation, live examples show the developer how the design system patterns will work in the final product. Several different examples for each pattern present the possibilities for individual design. \ac{SDS} users have the opportunity to share their creations and application of patterns below the documentation. \\

The big benefit of using SDS is the patterns. They simplify web accessibility compliance and leave enough room for individual user designs. The correct use of patterns is crucial for a successful integration of \ac{SDS} into a product.

\subsection{\ac{SDS} components}\label{sds_button}
An essential part of the implementation of a design system is the components. As described in chapter \ref{sds-component}, the design system builds its components using the Lit framework. This chapter presents an example of the button component of the \ac{SDS}. In addition, it shows a complete example of a component constructed with design tokens and documented with the Storybook.  \\
The components use TypeScript to take advantage of custom decorators. The decorators provided by Lit further simplify the boilerplate code for creating a web component. \\
\lstinputlisting[linerange={1-4},firstnumber=1,caption={Initialization of \ac{SDS} button component},label=ButtonInit]{../Code/src/components/button.component.ts}
In listing \ref{ButtonInit}, the \texttt{@customElement} decorator initializes the component by passing the tag name as a string and appending it to an \ac{ES6} class. For it to work correctly, the class must extend the Lit Element class. Finally, the web component registers itself, and users can use the component with the defined tag, in this case \texttt{<saas-button>}. \\
In order to see something when the web component just created is used, it must implement the render method. This method expects a \texttt{TemplateResult}. 
\lstinputlisting[linerange={19-26},firstnumber=19,caption={Rendering of \ac{SDS} button component},label=ButtonRender]{../Code/src/components/button.component.ts}
Lit element provides the import of an \texttt{html} string literal that constructs the \texttt{TemplateResult} expected by the render function. With this functionality, Lit provides an efficient way to respond to variable changes and intelligently update the \ac{DOM}. \\
Properties also use custom decorators and declare properties on web components by writing \texttt{@property} in front of a class attribute (line 19-20, Listing \ref{ButtonRender}). Thus, Lit Element implements change detection and add automatic type conversion. For a detailed description of the capabilities of this decorator, see the documentation. The property value of the created button web component changes and the constructed template string reacts to these changes and updates the displayed template accordingly. \\
Last but not least, the web component must use the defined design tokens. Since \ac{CSS} variables define these tokens, components consume them as follows: 
\lstinputlisting[linerange={5-20},firstnumber=5,caption={Styles of \ac{SDS} button component},label=ButtonStyles]{../Code/src/components/button.component.ts}
The Lit framework implements a static property of the \texttt{styles} of its classes. The component uses a literal \texttt{css} string to convert a string into the required \texttt{CSSResult} type. This way, it would be possible to consume design tokens via input properties. However, to manipulate design tokens in one place, the components of \ac{SDS} will use \ac{CSS} variables. It is possible to customize specific tokens throughout the design system by adapting them. Each component in the \ac{SDS} will use the defined tokens. Line 7-9 in listing \ref{ButtonStyles} shows an example of design tokens for the button component in the \ac{SDS}. \\
Finally, the button component is displayed in Storybook, the documentation tool for \ac{SDS}. Due to the limited time frame of this elaboration, the documentation is kept short. Figure \ref{storybook_button} is an example of the documentation of \ac{SDS} components. As in other mature design systems, the subsequent development of \ac{SDS} will add a full version of the documentation. \\
\begin{figure}[htbp]
    \centerline{\includegraphics[width=\linewidth]{images/storybook_button.png}}
    \caption{Example documentation of the \ac{SDS} button component}
    \label{storybook_button}
\end{figure}
At the moment, the documentation consists of a short description that briefly explains to the developer how and where he can use the button in his applications. In addition, a live example of the component is presented, automatically linked to the properties shown below. When the properties are changed, the example above is updated. This instant feedback gives developers or designers who want to use this component the chance to determine which configuration is most suitable for their use cases. \\
Inspecting the \ac{DOM} element of the button component, the element explorer looks like Figure \ref{button_element_explorer}. The new web component, the \texttt{saas-button}, is a valid \ac{DOM} element. A new shadow root is opened inside the button component. For this, the button component opens a new shadow root. The shadow root allows the web component to isolate styles and elements from the global document. The button component defines the elements used to display in this shadow root. Also, the code defines the styles for the button at the shadow \ac{DOM} level, so the rest of the \ac{DOM} tree never receives these styles. This technique helps to keep elements and styles separated and easy to understand. \\
\begin{figure}[htbp]
    \centerline{\includegraphics[height=100px]{images/button_element_explorer.png}}
    \caption{Inspection of \ac{SDS} button in element explorer}
    \label{button_element_explorer}
\end{figure}
It is crucial to declare that \ac{CSS} variables in the document's root are also available in the shadow \ac{DOM}. In contrast, style declarations, e.g., for the button element in the overall document, are not applied to \ac{DOM} elements in the shadow \ac{DOM}. \\
The \texttt{<slot>} element in the shadow \ac{DOM} is a default placeholder for anything inserted into the \texttt{<saas-button>} element when using the web component. The content is projected within the shadow \ac{DOM} and inserted in the place of the \texttt{<slot>} element. Slots make it possible to create nested elements with web components which is helpful in many different use cases.  \\

The example of a simple button component shows how to create and use the components of \ac{SDS}. From here, building all the different components needed to create patterns is trivial. The building is simple because patterns, as described, are a combination of components that applications reuse. The missing piece to using the \ac{SDS} in an application is integration, which the next chapter explains.
\subsection{Design System Integration}
The last step to complete the user story of the SDS is to integrate the system into an application. This chapter explains how to package the SDS and load it into a desired application. It is important that the integration is simple so as not to discourage developers from using the design system. \\

The build process is the first part that is important to understand the integration workflow. Webpack is the build tool used by SDS. As shown in Figure 3 in Chapter 1, the components of the design system are built using Typescript, including the Lit framework and using SCSS for styling. Since Typescript and SCSS are not supported by the browser, they must be processed beforehand. Therefore, Webpack comes into play to compile the code. \\
Webpack uses a configuration file, often called webpack.config.js, to describe the steps by which the code is compiled. This file describes rules that tell Webpack what to do with which files that come into the build pipeline. These rules for SDS look like this:\\
\lstinputlisting[linerange={23-41},firstnumber=23,caption={SDS Webpack rules},label=WebpackRules]{../Code/webpack.config.js}
The rules define regular expressions that are used to assign files with their extensions to the corresponding compilers, which are called loaders in the Webpack world. The loaders used here are build-in. But there are also custom ones that can be integrated into a build process. When using loaders, it is sufficient to include the loader string in the use property of a rule object. For example, in line 25-26, Listing \ref{WebpackRules}, the Typescript loader is matched with the regular expression Typescript to process Typescript files. The same pattern is used in line 34-35, Listing \ref{WebpackRules} to match SCSS files with the default style loader that comes with Webpack.